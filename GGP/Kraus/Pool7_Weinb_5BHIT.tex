\documentclass[letterpaper, 12pt]{article}

%%%%%%%%%%%%%%%%%%%%%%%%%%%%%
% DEFINITIONS
% Change those informations
% If you need umlauts you have to escape them, e.g. for an ü you have to write \"u
\gdef\mytitle{Ausarbeitung}
\gdef\mythema{Pool 7c/d}

\gdef\mysubject{GGP-Matura}
\gdef\mycourse{5BHIT 2015/16}
\gdef\myauthor{Michael Weinberger}

\gdef\myversion{1.0}
\gdef\mybegin{17. Mai 2016}
\gdef\myfinish{20. Mai 2016}

\gdef\mygrade{}
\gdef\myteacher{Betreuer: Kraus}
%
%%%%%%%%%%%%%%%%%%%%%%%%%%%%%

\input special/preamble.tex

\let\tempsection\section
\renewcommand\section[1]{\vspace{-0.3cm}\tempsection{#1}\vspace{-0.3cm}}
\WithSuffix\newcommand\section*[1]{\tempsection*{#1}}

\let\tempsubsection\subsection
\renewcommand\subsection[1]{\vspace{0cm}\tempsubsection{#1}\vspace{0cm}}

\let\tempsubsubsection\subsubsection
\renewcommand\subsubsection[1]{\vspace{0cm}\tempsubsubsection{#1}\vspace{0cm}}

\linespread{0.94}

\lhead{\mysubject}
\chead{}
\rhead{\bfseries\mythema}
\lfoot{\mycourse}
\cfoot{\thepage}
% Creative Commons license BY
% http://creativecommons.org/licenses/?lang=de
\rfoot{\ccby\hspace{2mm}\myauthor}
\renewcommand{\headrulewidth}{0.4pt}
\renewcommand{\footrulewidth}{0.4pt}

\begin{document}
\parindent 0pt
\parskip 6pt

\pagenumbering{Roman} 
\input{special/title}

\clearpage
\thispagestyle{empty}
\tableofcontents

\newpage
\pagenumbering{arabic}
\pagestyle{fancy}

%\vspace{-0.5cm}

\section{Pool 7c/d - Historische politische Entwicklungen und Konflikte sowie die Bedeutung für die Gegenwart \cite{buch}}

\subsection{Austrofaschismus 1933/34 bis 1938, Abkehr von der Demokratie}

\subsubsection{Heimwehr gegen Schutzbund - Der Kampf verlegt sich auf die Straße}

Bereits das Jahr 1927 wird zum ersten der sich danach häufenden 'Schicksalsjahre' Österreichs. Wobei das Jahr, zumindest für die bürgerlichen Parteien, unter gemischten Vorzeichen beginnt. Die Christlichsozialen haben zwar neun Sitze verloren, aber ihre Stellung als mandatsstärkste Partei noch ein letztes Mal behauptet. Am 19. Mai des Jahres kann der oftmalige Kanzler Ignaz Seipel sein fünftes - und letztes - Kabinett bilden. Die Mehrheit der Regierung ist durch die Koalition mit den kleinen Rechtsparteien abgesichert. Im Nationalrat droht ihr noch nicht jene Gefahr, die sechs Jahre später zum Sturz der demokratischen Republik beitragen wird. Doch die Spannungen im Lande — das heißt im 'außerparlamentarischen Raum' — haben sich beträchtlich verschärft. \\
Schon Anfang des Jahres war es in den östlichen Bundesländern immer
häufiger zu gewalttätigen Zusammenstößen zwischen Heimatschützern und Frontkämpfern (die Trennungslinie zwischen den
beiden ist hier nicht so klar wie im Westen) auf der einen und
dem Republikanischen Schutzbund auf der anderen Seite gekommen. \\ Doch es kommt zu keiner Beruhigung. Am 14. Juli spricht ein offensichtlich politisch beeinflusstes Gericht einige Frontkämpfer nicht nur von der Mordanklage, sondern auch von der Beschuldigung der Notwehrüberschreitung frei. Die Empörung im sozialdemokratischen Lager ist ungeheuer und wird durch einen Brandartikel im Parteiorgan 'Arbeiterzeitung' noch angeheizt.
Am folgenden Tag eskalieren Protestdemonstrationen in Wien zu schweren Ausschreitungen, die im Brand des Justizpalastes gipfeln. Der sozialdemokratischen Führung war es nicht gelungen, die Massen zu kontrollieren. Nicht einmal die Feuerwehr kann, trotz erregter Appelle von Wiens 'rotem' Bürgermeister Seitz, den Brandherd erreichen. Außer Kontrolle — oder war es, wie die Linke später mutmaßte, mit voller Absicht? — gerät danach auch die Gegenaktion der Polizei. Schließlich zählt man 90 Tote und 600 Verletzte, zumeist aus den Reihen der Demonstranten. \\ Der Justizpalast soll nicht zufällig zum Ziel der Aufrührer geworden sein. Vielmehr sollten wichtige Dokumente der Justizverwaltung zerstört werden. Auf der anderen Seite wird der brutale Polizeieinsatz nicht als spontane Überreaktion, sondern als gezielter Versuch gewertet, die politischen Gegner durch einen Kraftakt einzuschüchtern. Wahrscheinlich dürfte die historische Wahrheit — wie so oft - in der Mitte liegen. Wie dem auch sei, die Gemüter in Wien hatten sich noch lange nicht beruhigt. \\
Noch wesentlich schwerer wog ein 'äußeres Ereignis'. Der 'schwarze Freitag' (29. Oktober 1929) an der New Yorker Börse löste innerhalb kurzer Zeit eine Weltwirtschaftskrise aus. Diese wirkte sich in Deutschland besonders stark aus, das sich noch immer mit Reparationszahlungen herumzuschlagen hatte, auf die man gegenüber Österreich bereits im Jänner 1929 verzichtet hatte. Unmittelbare Folge der Wirtschaftskrise in Deutschland war das dramatische Aufkommen der Nationalsozialisten. \\ Bei den Reichstagswahlen (24. 9. 30) wurden die Nationalsozialisten (hinter den Sozialdemokraten) zur zweitstärksten Partei der Weimarer Republik. Die Folgen für Österreich waren vorprogrammiert. Die Nazis, die bei allen bisherigen Wahlen zum Nationalrat, trotz beginnender Wirtschaftskrise, noch kein einziges Mandat errungen hatten, nutzten die unsicheren Verhältnisse aus und wurden immer radikaler. Doch auch im bürgerlich-konservativen Lager setzte eine weitere Radikalisierung ein, die sich im 'Korneuburger Eid' der Heimwehren manifestiert hatte. Die Heimwehren legten sich forthin auf einen faschistischen Kurs nach italienischem Muster fest. Aus dem Kampf 'Links gegen Rechts' beziehungsweise dem Stellvertreterkrieg Heimwehr gegen Schutzbund, wurde nun dank des Auftretens der Nazis ein Kampf jeder gegen jeden. Es kam zu Umwälzungen bei den Heimwehren, Ernst Rüdiger Starhemberg war fortan der neue starke Mann. \\ Entscheidend für Starhembergs Hinwendung zum alten Erzfeind Italien war jedoch die militärische Hilfe in diesem Bürgerkrieg. Mussolini und, in geringerem Maße, Ungarns Gömbös ließen den Heimwehren beachtliche Ausmaße zukommen. Diese Privatarmee hatte zu jenem Zeitpunkt einen beträchtlichen Aktivstand von 50.000 Mann. Sie war also eine militärische Kraft welche jene des kleinen Bundesheeres bei weitem übertraf. \\
Am 18. März 1931 war der niederösterreichische Bauernbunddirektor Engelbert Dollfuß, ein Neuling in der Politik, als Landwirtschaftsminister ins Kabinett eingetreten. Der nächste österreichische Regierungschef war wieder ein Christichsozialer, Karl Buresch. Und neuerlich setzte die Regierung auf eine wacklige Koalition der Christlichsozialen mit dem Landbund und diesmal auch den Großdeutschen.

\subsubsection{Das Parlament hat sich ausgeschaltet - Sozialistischer Aufstand und Nazi-Terror}

Am 20. Mai 1931 wurde eine neue bürgerliche Koalition aus Christlichsozialen, Landbund und Heimatblock auf die Beine gestellt, mit einer Mehrheit von sage und schreibe einer Stimme. Das Bemerkenswerte war jedoch die Person des neuen Bundeskanzlers - der bisherige Landwirtschaftsminister Engelbert Dollfuß. Vor seinem Eintritt in das Kabinett hatte der neue Kanzler, aus ärmlichen Verhältnissen stammend, niemals ein höheres politisches Amt eingenommen und unterschied sich schon dadurch von allen seinen Vorgängern aus dem alten christlichdemokratischen Lager. Ein neuer Mann genügte indessen nicht für einen neuen Anfang. \\ Auch das Kabinett Dollfuß wurde von der Bevölkerung als eine weitere Übergangsregierung angesehen. Doch da hatten sie sich geirrt. Sie hatten nämlich nicht mit der starken Persönlichkeit des Regierungschefs gerechnet. Und diese Unkenntnis war nicht überraschend. In der Öffentlichkeit war Dollfuß weitgehend unbekannt. Das einzig Auffällige an ihm war seine kleine Körpergröße, 1.60m. Bekanntlich sind es da wie dort immer die Kleinen, die womöglich etwas kompensieren müssen. \\
Dollfuß versuchte die parlamentarische Maschinerie so weit wie möglich auszuschalten. Den Weg dazu wies ihm sein Verfassungsexperte Robert Hecht. Dieser 'entdeckte' gleichsam das Kriegswirtschaftliche Ermächtigungsgesetz von 1917, welches der damaligen Regierung Befugnisse einräumte, sich über das Parlament hinwegzusetzen. Natürlich hatte dieses Gesetz 1918 seinen eigentlichen Sinn verloren, es war aber niemals formell abgeschafft worden. Die Sozialdemokraten hatten dies mehrmals gefordert — ohne allzu großen Nachdruck —, waren aber nicht durchgedrungen. So war das Gesetz formell noch immer in Kraft, zumindest gemäß der Rechtsauffassung von Dollfuß und dessen Berater Hecht. Es wurde übrigens erst nach dem Zweiten Weltkrieg aus den Statuten gelöscht. \\
Kurt Schuschnigg war von Dollfuß als Justizminister übernommen worden, zeitweilig agierte er auch als Unterrichtsminister. Gleichsam als Rechtswahrer innerhalb des Kabinetts hätte er eigentlich den offenkundigen Missbrauch des Ermächtigungsgesetzes verhindern oder zumindest missbilligen müssen. Er tat jedoch nichts dergleichen, unterstützte sogar den immer autoritäreren Kurs der Regierung. \\ Das Schicksal der Regierung stand wegen der sehr knappen Mehrheit nun auf des Messers Schneide. Und die Minister wussten es. Um einen Sturz zu vermeiden, wurde am 1. 10. 1932 das Kriegswirtschaftliche Ermächtigungsgesetz erstmals in einer substantiellen politischen Frage, nämlich in Zusammenhang mit der Haftbarmachung der Spitzen der Credit-Anstalt für die Verluste der Bank, angewendet. Um nicht sofort zu stürzen, war Dollfuß mehr denn je von seinen unberechenbaren Heimwehr—Partnern abhängig. \\ Deren 'starker Mann' Fürst Starhemberg hatte sich geweigert, nach seinem kurzen ersten 'Gastspiel' als Innenminister, neuerlich selbst in die Regierung einzutreten, um nicht persönlich für eventuelle Konflikte oder unpopuläre Maßnahmen mitverantwortlich gemacht zu werden. Er war aber bereit, einen Kontakt zwischen Dollfuß und Mussolini, seinem neuen Freund, herzustellen — ohne zu wissen (oder vielleicht mit Absicht?), dass diese Beziehung, die sich bald in ein Abhängigkeitsverhältnis verwandeln sollte, Österreich direkt in den Bürgerkrieg führen würde.
\subsubsection{Zunehmender Aufstieg der Nationalsozialisten}
Nun aber müssen wir den Blick ins Ausland wenden, weil sich die Ereignisse in Deutschland überstürzen. Am 10. April 1932 hatte die deutsche Demokratie ihren letzten Sieg erfochten. Staatspräsident Hindenburg war mit 53 Prozent der Stimmen für eine zweite Amtszeit wiedergewählt worden während Hitler deutlich im geschlagenen Feld landete. Den Sieg verdankte der alte Feldmarschall in erster Linie dem Einsatz von Reichskanzler Brüning, doch Undank war dessen Lohn. Er fiel einer Verschwörung um den greisen Hindenburg zum Opfer und trat am 30. Mai des Jahres zurück.
Sein Nachfolger wurde der zu jenem Zeitpunkt weniger als Politiker denn als Reiter bekannte Franz von Papen, ein Zentrumspolitiker wie Brüning, der das höchste Regierungsamt gegen den Wunsch seiner Partei übernahm. Papen ließ sofort den Reichstag auflösen und setzte Neuwahlen für den 31. Juli an, die den Nationalsozialisten ihren bisher größten Sieg, aber nicht annähernd die absolute Mehrheit brachten, da Hitlers Gewinne in erster Linie auf Kosten der Deutschnationalen gingen. Papen bot Hitler daraufhin das Amt des Vizekanzlers an, doch dieser war entschlossen, aufs Ganze zu gehen. \\
Im Reichstag erleidet der Kanzler schon bei der ersten Sitzung eine vernichtende Niederlage und lasst danach das Parlament von Hindenburg zum zweiten Mal auflösen. Die nächsten Wahlen am 6. November bringen für die Nazis erstmals einen Rückschlag, aber sie bleiben weiter stärkste Einzelfraktion. Infolge der Wahl kommt es zu einem Machtgerangel zwischen Papen \& Schleicher, wo jeder an dem Sturz des jeweils anderen arbeitet. Die österreichische Regierung hat natürlich die Entwicklung in Deutschland mit wachsender Besorgnis verfolgt. \\
Dollfuß blickt jedoch nicht nur nach Deutschland. Von rechts und links bedrängt, sucht er engeren Kontakt zu Italien. Er bittet Mussolini um Waffen, um sich besser verteidigen zu können. Das passt genau in Mussolinis Pläne, und es fällt ihm leicht, den Wunsch zu erfüllen, weil er genügend österreichische Beutewaffen aus den letzten Tagen des Krieges besitzt — etwas ramponiert, aber noch immer benutzbar. \\ Der Handel darf natürlich nicht bekannt werden, da er gegen die Bestimmungen von St. Germain verstößt, und so einigt man sich auf eine 'Umleitung' Das Kriegsgerät soll mit falscher Deklaration von Italien über Österreich nach Ungarn geschickt werden. Beim Zwischenstopp in Österreich sollen die Waffen instandgesetzt und ein Teil soll an die Heimwehren abgezweigt werden. \\ Dieses 'Service' brachte immerhin um die 50.000 Gewehre und 200 Maschinengewehre ins Land. \\
Die Transaktion wird jedoch am 8. Jänner 1933 in der 'Arbeiterzeitung' auf Grund eines Tipps enthüllt. Das Ergebnis ist eine totale Blamage für die Regierung, und auch Mussolini gerät ins Kreuzfeuer der britischen und französischen Kritik. Dollfuß betrachtet die Affäre als Kriegserklärung der Linken, und Mussolini drängt ihn, mit dem demokratischen Spuk ein für allemal aufzuräumen. \\
Die Aufregung legt sich indessen bald, weil Hitler am 30. Jänner 1933 die Macht in Deutschland übernimmt, ein Schritt, der die Weltgeschichte grundlegend verändern wird, wiewohl dies noch niemand zu jenem Zeitpunkt ahnt. \\
Die österreichischen Nazis sehen jedoch ihr Ziel in greifbare Nähe gerückt und verstärken den Terror auf der Straße, während sie gleichzeitig scheinheilig Neuwahlen fordern. Intern haben sie den Anschluss längst vollzogen. Für sie ist Österreich ein Gau des Reiches und ihr Führer ist konsequenterweise ein Reichsdeutscher.
\subsubsection{'Selbstausschaltung' der Volksvertretung}
Die Regierung unter Dollfuß entscheidet sich weiter für Härte, will auch keine Neuwahlen riskieren, da sie sonst herbe Verluste hinnehmen würde. Der Nationalrat scheint zumindest momentan gelähmt durch den durchgehenden Lagerkampf. Dollfuß hat wahrscheinlich nur auf diesen Eklat gewartet, um das Parlament vollständig auszuschalten. Er und seine Anhänger sprechen jedenfalls von einer 'Selbstausschaltung' der Volksvertretung, die Sozialdemokraten von einer gezielten Ausschaltung der gewählten Repräsentanten des österreichischen Volkes. Auch heute noch gehen die beiden traditionellen Lager in ihrer Interpretation der Ereignisse vom 4. März 1933 diametral auseinander. \\
Die von Schuschnigg angeregten autoritären Maßnahmen werden sofort in die Tat umgesetzt. Neben der Einschränkung der Pressefreiheit wird auch ein Aufmarsch- und Versammlungsverbot verhängt. Die Oppositionsparteien suchten indessen einen Ausweg aus der Krise. zu der sie ja wohl selbst, durch das Verhalten ihrer Parlamentspräsidenten, beigetragen hatten. Der letzte amtierende Nationalratsvorsitzende, der Großdeutsche Sepp Straffner, nahm seine Demission zurück und berief eine weitere Sitzung des Parlaments für den 15. März ein. \\
Im Ministerrat wandte sich gerade Schuschnigg besonders heftig gegen diesen Schachzug. Das Kabinett müsse gegen diesen Schritt öffentlich Stellung beziehen, sonst desavouiere es sich selbst: 'Dabei wäre zu unterstreichen, dass die Regierung auf dem Boden der Verfassung stehe und Straffner derjenige sei, der die Verfassung, gelinde gesagt, irrig auslege und sich ein Recht usurpiere.' (Auszug aus dem Protokoll.) \\
So geschieht es auch. Als sich die Oppositionsabgeordneten am 15. Mai in - oder vor - dem Hohen Haus einfinden, werden diejenigen, die in das Gebäude gelangt sind, von der Polizei gewaltsam vertrieben. Die Präsidententribüne wird von der Exekutive besetzt, um zu verhindern, dass ein Abgeordneter das Präsidium übernehme. Nun folgt Schlag auf Schlag. Nachdem die Regierung bereits den sozialdemokratischen Wiener Polizeipräsidenten abgesetzt hat, wird am 31. März der Republikanische Schutzbund aufgelöst, und Dollfuß beschließt kurz danach die Aufstellung 'freiwilliger Assistenzkörper‚ wodurch die Heimwehren durch die Hintertür legalisiert werden. \\
International erregen diese Maßnahmen nur wenig Aufsehen. Sie muten geradezu milde an, verglichen mit dem, was sich gleichzeitig im Deutschen Reich abspielt, wo der unter schwerstem Terror 'gewählte' neue Reichstag dem 'Führer' diktatorische Vollmachten eingeräumt hat. \\
Bereits am 13. März ist Dollfuß zu seinem ersten Treffen nach der totalen Machtübernahme nach Rom gefahren, wo er von Mussolini mit offenen Armen aufgenommen wird, der ihm zu den getroffenen Maßnahmen gratuliert. Auch auf persönlicher Ebene entwickelt sich ein Nahverhältnis, das sich für Österreich noch schicksalhaft auswirken wird. Überdies bietet der Duce nicht nur militärische, sondern auch wirtschaftliche Hilfe, die in den folgenden Jahren einen Pfeiler der österreichischen Wirtschaftspolitik bilden wird. \\
Kurz nach der Dollfußreise halten Österreichs Sozialdemokraten in einer kampfbetonten Stimmung ihre letzte Landeskonferenz ab. Der Kanzler reagiert darauf mit einem Streikverbot. Bei 365.000 Arbeitslosen kann er sich das leisten. Er wagt es sogar, den traditionellen Maiaufmarsch der Sozialisten durch militärische Absperrmaßnahmen zu verhindern. Die 'Roten' reagieren mit einem 'Bummelkorso', den sie nunmehr regelmäßig wiederholen. Von einer Machtdemonstration ist dies aber weit entfernt. Die rechte 'Opposition', sofern man sie noch so nennen kann, begrüßt die gewaltsamen Maßnahmen gegen die Linke. Und dies verleitet den Kanzler zu einem Fehler, zum letzten Mal gestattet er einen freien Wahlgang — zum Innsbrucker Gemeinderat. Das Ergebnis ist eine katastrophale Niederlage für die in Tirol so siegesgewohnten 'Schwarzen'. Die
Nationalsozialisten werden mit 40 Prozent der Stimmen stärkste
Partei, ein Schlag, der nicht nur die Tiroler Volkspartei, sondern
auch den Wahlinnsbrucker Schuschnigg besonders hart trifft. Zur Ehrenrettung der Christlichsozialen muss gesagt werden, dass der autoritäre, undemokratische Kurs von dem damals noch immer aktiven demokratischen Flügel innerhalb der Partei scharf missbilligt wird. Beim Bundesparteitag setzt sich jedoch der autoritäre Flügel durch und spricht Dollfuß vollstes Vertrauen aus. \\
Dieser Rechtsruck manifestiert sich schon wenige Tage später in einer Umbildung des Kabinetts Dollfuß. Diese Maßnahmen haben weitere erhebliche Machtverstärkungen der Rechten zur Folge, und auch die Stellung der Heimwehr wird aufgewertet. \\
Die Sozialdemokraten holen zu einem letzten Schlag aus, um das Parlament wiederzubeleben, und kontaktiert den bis dahin noch bestehenden Verfassungsgerichtshof. Dollfuß kontert schnell, und beruft die Richter aus dem Senat ab. Der Gerichtshof, als letzte Schlichtungsstelle, ist damit ebenso lahmgelegt.
\subsubsection{Errichtung der Vaterländischen Front}
Dollfuß arbeitete konsequent am Ausbau seines neuen Systems. Am 20. Mai 1933 hebt er die 'Vaterländische Front' aus der
Taufe, von der vorerst niemand genau weiß. was sie darstellen
soll. Immerhin gibt es ja noch die alten Parteien. Dennoch ist es
gerade Schuschnigg, der die Gründung der 'VF' in verschiedenen Ansprachen begrüßt. Er will sie nicht als eine Art faschistische Partei angesehen wissen. sondern als Sammelbewegung aller staatstragenden Kräfte. Sie entwickelt sich jedoch sehr langsam und dringt eigentlich nie als Alternative zu den Parteien in
das Bewusstsein der Österreicher ein. Noch ist die Konkurrenz zu
groß und ihre Konzepte sind zu vage. Nur die Kommunistische
Partei wird eine Woche später als erste politische Gruppierung
aufgelöst.
Da die Großdeutsche Partei als politische Kraft de facto ausgeschieden ist und eine Versöhnung mit den Sozialdemokraten
nunmehr unmöglich erscheint. glaubt sich Dollfuß gezwungen.
trotz der immer aggressiveren Politik des Dritten Reiches mit
den Nazis, sowohl in Österreich als auch jenseits der Grenze
Kontakt aufzunehmen. \\
Die Forderungen der Nazis waren jedoch kaum zu erfüllen, sie zielten eher auf einen politischen Umsturz nach ihren Vorstellungen ab. Obwohl die Heimwehr in den Grundzügen bzw. in Ansätzen die gleichen Ideen hat, verlaufen Gespräche erfolglos. Nach diesem Gespräch verschärfen die Nazis den Druck, sowohl innerhalb des Landes, als auch außerhalb. \\
Sanktionen wurden verhängt, die die Fremdenverkehrswirtschaft durch abstruse Richtlinien derart zerstört hatte. Zwischen der Regierung und den Nazis herrscht ein offener, wenn auch nicht erklärter Krieg. Noch im selben Jahr wurde nach Anschlägen auf Katholiken die NSDAP verboten. Auch der stetig wachsende Antisemitismus hat die Juden in wachsende Besorgnis versetzt. Angesichts der deutschen Wirtschaftsblockade war Österreich auf Italien angewiesen. Ein weiteres Treffen mit dem Duce brachte poltische Absicherungen und eine Erhöhung der Importe. \\
Derart abgesichert, fühlte sich Dollfuß politisch gestärkt. Bei einer Massenversammlung erklärte er sein politisches Ziel: 'Die Errichtung eines sozialen, christlichen, deutschen Staates Österreichs auf ständischer Grundlage unter starker autoritärer Führung' gegen Marxismus, kapitalistische Weltordnung, Nationalsozialismus und Parteienherrschaft.
\subsubsection{Folgen bis zum Anschluss Österreichs 1938}
Sein Vorhaben sollte nicht lang halten, im Juli 1934 versuchten die weitgehend im Illegalen arbeitenden österreichische Nationalsozialisten im sogenannten Juliputsch die austrofaschistische Regierung abzusetzen und die Macht an sich zu reißen. Zwar misslang der Putschversuch, aber sie ermordeten den Kanzler Engelbert Dollfuß. Ihm folgte der vormalige Justizminister Kurt Schuschnigg als Bundeskanzler. Auch dieser stand dem Nationalsozialismus und dem Anschluss an das Deutsche Reich kritisch gegenüber. Nur unter Druck schloss er 1936 das Juliabkommen mit dem Deutschen Reich, in dem er Zugeständnisse an die Nationalsozialisten machte, dafür aber von Adolf Hitler die Unabhängigkeit Österreichs zugesichert bekam. In der Folge wurden 17.000 österreichische Nationalsozialisten amnestiert. Die Nationalsozialisten erhielten durch diese Entwicklungen zusehends Auftrieb. \\
Am 12. Februar 1938 nötigte Hitler Schuschnigg zum so genannten Berchtesgadener Abkommen, das die schrittweise Machtübernahme der Nationalsozialisten fortsetzte. Der Nationalsozialist Arthur Seyß-Inquart, der 1937 in den Staatsrat berufen worden war, wurde Innen- und Sicherheitsminister in der Schuschnigg-Regierung. Schuschnigg setzte am 9. März als letzten Versuch, Österreichs Unabhängigkeit zu bewahren, für den 13. März eine Volksabstimmung über Österreichs Unabhängigkeit an. Daraufhin wurde er jedoch von Hitler unter Drohung militärischen Eingreifens zur Abdankung zugunsten Seyß-Inquarts gezwungen. Am 12. März erfolgte der Einmarsch der Deutschen, ohne auf militärischen Widerstand zu treffen, und unter dem Jubel zahlreicher Österreicher. Die Abstimmung fand nicht mehr statt.
\\
Schuschnigg wurde später von den Nationalsozialisten in einem Konzentrationslager in Deutschland inhaftiert. Auch wenn er dort noch eine Vorzugsbehandlung erhielt, entging er dem Tod nur aufgrund einer Befreiungsaktion 

\clearpage
\bibliographystyle{unsrt}
\bibliography{Pool7_Weinb_5BHIT}

\end{document}
