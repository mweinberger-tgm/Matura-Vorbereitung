\documentclass[letterpaper, 12pt]{article}

%%%%%%%%%%%%%%%%%%%%%%%%%%%%%
% DEFINITIONS
% Change those informations
% If you need umlauts you have to escape them, e.g. for an ü you have to write \"u
\gdef\mytitle{Ausarbeitung}
\gdef\mythema{Pool 7c/d}

\gdef\mysubject{GGP-Matura}
\gdef\mycourse{5BHIT 2015/16}
\gdef\myauthor{Michael Weinberger}

\gdef\myversion{1.0}
\gdef\mybegin{17. Mai 2016}
\gdef\myfinish{20. Mai 2016}

\gdef\mygrade{}
\gdef\myteacher{Betreuer: Kraus}
%
%%%%%%%%%%%%%%%%%%%%%%%%%%%%%

\input special/preamble.tex

\let\tempsection\section
\renewcommand\section[1]{\vspace{-0.3cm}\tempsection{#1}\vspace{-0.3cm}}
\WithSuffix\newcommand\section*[1]{\tempsection*{#1}}

\let\tempsubsection\subsection
\renewcommand\subsection[1]{\vspace{0cm}\tempsubsection{#1}\vspace{0cm}}

\let\tempsubsubsection\subsubsection
\renewcommand\subsubsection[1]{\vspace{0cm}\tempsubsubsection{#1}\vspace{0cm}}

\linespread{0.94}

\lhead{\mysubject}
\chead{}
\rhead{\bfseries\mythema}
\lfoot{\mycourse}
\cfoot{\thepage}
% Creative Commons license BY
% http://creativecommons.org/licenses/?lang=de
\rfoot{\ccby\hspace{2mm}\myauthor}
\renewcommand{\headrulewidth}{0.4pt}
\renewcommand{\footrulewidth}{0.4pt}

\begin{document}
\parindent 0pt
\parskip 6pt

\pagenumbering{Roman} 
\input{special/title}

\clearpage
\thispagestyle{empty}
\tableofcontents

\newpage
\pagenumbering{arabic}
\pagestyle{fancy}

%\vspace{-0.5cm}

\section{Pool 7c/d - Historische politische Entwicklungen und Konflikte sowie die Bedeutung für die Gegenwart \cite{buch}}

\subsection{Austrofaschismus 1933/34 bis 1938, Abkehr von der Demokratie}

\subsubsection{Heimwehr gegen Schutzbund - Der Kampf verlegt sich auf die Straße}

Bereits das Jahr 1927 wird zum ersten der sich danach häufenden 'Schicksalsjahre' Österreichs. Wobei das Jahr, zumindest für die bürgerlichen Parteien, unter gemischten Vorzeichen beginnt. Die Christlichsozialen haben zwar neun Sitze verloren, aber ihre Stellung als mandatsstärkste Partei noch ein letztes Mal behauptet. Am 19. Mai des Jahres kann der oftmalige Kanzler Ignaz Seipel sein fünftes - und letztes - Kabinett bilden. Die Mehrheit der Regierung ist durch die Koalition mit den kleinen Rechtsparteien abgesichert. Im Nationalrat droht ihr noch nicht jene Gefahr, die sechs Jahre später zum Sturz der demokratischen Republik beitragen wird. Doch die Spannungen im Lande — das heißt im 'außerparlamentarischen Raum' — haben sich beträchtlich verschärft. Schon
Anfang des Jahres war es in den östlichen Bundesländern immer
häufiger zu gewalttätigen Zusammenstößen zwischen Heimatschützern und Frontkämpfern (die Trennungslinie zwischen den
beiden ist hier nicht so klar wie im Westen) auf der einen und
dem Republikanischen Schutzbund auf der anderen Seite gekommen. Doch es kommt zu keiner Beruhigung. Am 14. Juli spricht ein offensichtlich politisch beeinflusstes Gericht einige Frontkämpfer nicht nur von der Mordanklage, sondern auch von der Beschuldigung der Notwehrüberschreitung frei. Die Empörung im sozialdemokratischen Lager ist ungeheuer und wird durch einen Brandartikel im Parteiorgan 'Arbeiterzeitung' noch angeheizt.
Am folgenden Tag eskalieren Protestdemonstrationen in Wien zu schweren Ausschreitungen, die im Brand des Justizpalastes gipfeln. Der sozialdemokratischen Führung war es nicht gelungen, die Massen zu kontrollieren. Nicht einmal die Feuerwehr kann, trotz erregter Appelle von Wiens 'rotem' Bürgermeister Seitz, den Brandherd erreichen. Außer Kontrolle — oder war es, wie die Linke später mutmaßte, mit voller Absicht? — gerät danach auch die Gegenaktion der Polizei. Schließlich zählt man 90 Tote und 600 Verletzte, zumeist aus den Reihen der Demonstranten. Der Justizpalast soll nicht zufällig zum Ziel der Aufrührer geworden sein. Vielmehr sollten wichtige Dokumente der Justizverwaltung zerstört werden. Auf der anderen Seite wird der brutale Polizeieinsatz nicht als spontane Überreaktion, sondern als gezielter Versuch gewertet, die politischen Gegner durch einen Kraftakt einzuschüchtern. Wahrscheinlich dürfte die historische Wahrheit — wie so oft - in der Mitte liegen. Wie dem auch sei, die Gemüter in Wien hatten sich noch lange nicht beruhigt. \\
Noch wesentlich schwerer wog ein 'äußeres Ereignis'. Der 'schwarze Freitag' (29. Oktober 1929) an der New Yorker Börse löste innerhalb kurzer Zeit eine Weltwirtschaftskrise aus. Diese wirkte sich in Deutschland besonders stark aus, das sich noch immer mit Reparationszahlungen herumzuschlagen hatte, auf die man gegenüber Österreich bereits im Jänner 1929 verzichtet hatte. Unmittelbare Folge der Wirtschaftskrise in Deutschland war das dramatische Aufkommen der Nationalsozialisten. Bei den Reichstagswahlen (24. 9. 30) wurden die Nationalsozialisten (hinter den Sozialdemokraten) zur zweitstärksten Partei der Weimarer Republik. Die Folgen für Österreich waren vorprogrammiert. Die Nazis, die bei allen bisherigen Wahlen zum Nationalrat, trotz beginnender Wirtschaftskrise, noch kein einziges Mandat errungen hatten, nutzten die unsicheren Verhältnisse aus und wurden immer radikaler. Doch auch im bürgerlich-konservativen Lager setzte eine weitere Radikalisierung ein, die sich im 'Korneuburger Eid' der Heimwehren manifestiert hatte. Die Heimwehren legten sich forthin auf einen faschistischen Kurs nach italienischem Muster fest. Aus dem Kampf 'Links gegen Rechts' beziehungsweise dem Stellvertreterkrieg Heimwehr gegen Schutzbund, wurde nun dank des Auftretens der Nazis ein Kampf jeder gegen jeden. Es kam zu Umwälzungen bei den Heimwehren, Ernst Rüdiger Starhemberg war fortan der neue starke Mann. Entscheidend für Starhembergs Hinwendung zum alten Erzfeind Italien war jedoch die militärische Hilfe in diesem Bürgerkrieg. Mussolini und, in geringerem Maße, Ungarns Gömbös ließen den Heimwehren beachtliche Ausmaße zukommen. Diese Privatarmee hatte zu jenem Zeitpunkt einen beträchtlichen Aktivstand von 50.000 Mann. Sie war also eine militärische Kraft welche jene des kleinen Bundesheeres bei weitem übertraf. \\
Am 18. März 1931 war der niederösterreichische Bauernbunddirektor Engelbert Dollfuß, ein Neuling in der Politik, als Landwirtschaftsminister ins Kabinett eingetreten. Der nächste österreichische Regierungschef war wieder ein Christichsozialer, Karl Buresch. Und neuerlich setzte die Regierung auf eine wacklige Koalition der Christlichsozialen mit dem Landbund und diesmal auch den Großdeutschen.

\subsubsection{Das Parlament hat sich ausgeschaltet - Sozialistischer Aufstand und Nazi-Terror}

Am 20. Mai 1931 wurde eine neue bürgerliche Koalition aus Christlichsozialen, Landbund und Heimatblock auf die Beine gestellt, mit einer Mehrheit von sage und schreibe einer Stimme. Das Bemerkenswerte war jedoch die Person des neuen Bundeskanzlers - der bisherige Landwirtschaftsminister Engelbert Dollfuß. Vor seinem Eintritt in das Kabinett hatte der neue Kanzler, aus ärmlichen Verhältnissen stammend, niemals ein höheres politisches Amt eingenommen und unterschied sich schon dadurch von allen seinen Vorgängern aus dem alten christlichdemokratischen Lager. Ein neuer Mann genügte indessen nicht für einen neuen Anfang. Auch das Kabinett Dollfuß wurde von der Bevölkerung als eine weitere Übergangsregierung angesehen.

\clearpage
\bibliographystyle{unsrt}
\bibliography{Pool7_Weinb_5BHIT}

\end{document}
