\documentclass[letterpaper, 12pt]{article}

%%%%%%%%%%%%%%%%%%%%%%%%%%%%%
% DEFINITIONS
% Change those informations
% If you need umlauts you have to escape them, e.g. for an ü you have to write \"u
\gdef\mytitle{Ausarbeitung}
\gdef\mythema{Pool 2a}

\gdef\mysubject{GGP-Matura}
\gdef\mycourse{5BHIT 2015/16}
\gdef\myauthor{Michael Weinberger}

\gdef\myversion{1.0}
\gdef\mybegin{17. Mai 2016}
\gdef\myfinish{20. Mai 2016}

\gdef\mygrade{}
\gdef\myteacher{Betreuer: Kraus}
%
%%%%%%%%%%%%%%%%%%%%%%%%%%%%%

\input special/preamble.tex

\let\tempsection\section
\renewcommand\section[1]{\vspace{-0.3cm}\tempsection{#1}\vspace{-0.3cm}}
\WithSuffix\newcommand\section*[1]{\tempsection*{#1}}

\let\tempsubsection\subsection
\renewcommand\subsection[1]{\vspace{0cm}\tempsubsection{#1}\vspace{0cm}}

\let\tempsubsubsection\subsubsection
\renewcommand\subsubsection[1]{\vspace{0cm}\tempsubsubsection{#1}\vspace{0cm}}

\linespread{0.94}

\lhead{\mysubject}
\chead{}
\rhead{\bfseries\mythema}
\lfoot{\mycourse}
\cfoot{\thepage}
% Creative Commons license BY
% http://creativecommons.org/licenses/?lang=de
\rfoot{\ccby\hspace{2mm}\myauthor}
\renewcommand{\headrulewidth}{0.4pt}
\renewcommand{\footrulewidth}{0.4pt}

\begin{document}
\parindent 0pt
\parskip 6pt

\pagenumbering{Roman} 
\input{special/title}

\clearpage
\thispagestyle{empty}
\tableofcontents

\newpage
\pagenumbering{arabic}
\pagestyle{fancy}

%\vspace{-0.5cm}

\section{Pool 2a - Globale Entwicklungstrends \cite{buch}}

\subsection{Kolonialmacht Großbritannien und deren Zerfall nach 1945}

Obwohl Großbritannien mit dem Empire den Zweiten Weltkrieg als eine der Hauptmächte der Anti-Hitler-Koalition 1945 erfolgreich beenden konnte, hatte der Konflikt tiefgreifende Auswirkungen. Europa, ein Kontinent, der die Welt mehrere Jahrhunderte lang dominiert hatte, lag buchstäblich in Trümmern. Die nunmehr dominierenden Weltmächte USA und Sowjetunion hatten ihren Machtbereich enorm ausdehnen können. \\ In einer Reihe von Staaten wurden Besatzungstruppen stationiert, ihr politisches System eingeführt und Militärstützpunkte errichtet. Sie stiegen folglich zu globalen Supermächten auf. Großbritannien wiederum hatte riesige Schulden angehäuft und entging 1946 nur knapp dem Staatsbankrott, nicht zuletzt dank einer US-Anleihe in Höhe von 3,5 Milliarden Dollar. \\
Zur selben Zeit gewannen antikolonialistische Bewegungen an Bedeutung. Die Situation wurde durch die wachsenden Spannungen im Kalten Krieg zwischen den Vereinigten Staaten und der Sowjetunion weiter verkompliziert. Beide Staaten lehnten den europäischen Kolonialismus ab, wenngleich bei den Amerikanern und Westeuropäern der Antikommunismus weitaus stärker ausgeprägt war als der Antiimperialismus und die Briten deshalb weiterhin Unterstützung erhielten. Das Ende des Britischen Weltreichs war absehbar und Großbritannien versuchte eine Politik des friedlichen Rückzugs aus den Kolonien, was nicht immer gelang. Ziel war es einerseits die Staatsgewalt an stabile antikommunistische Regierungen zu übertragen und andererseits durch stabile wirtschaftliche Beziehungen den britischen Siedlern weiterhin eine sichere Heimat zu garantieren. In manchen ehemaligen Kolonien Afrikas etablierte sich jedoch ein afrikanischer Sozialismus, wie z. B. in Sambia oder Tansania. Andere Staaten wie Frankreich oder Portugal, führten teilweise kostspielige und letztlich erfolglose Kriege, um ihre Kolonialreiche zu retten. Zwischen 1945 und 1965 nahm die Zahl der Menschen, die außerhalb des Vereinigten Königreichs unter britischer Herrschaft standen, von 700 Millionen auf fünf Millionen ab (davon drei Millionen in Hongkong).

\subsubsection{Vorreiter Indien}

Vor dem Zweiten Weltkrieg hieß es noch 'Die Sonne im britischen Empire geht nie unter'. Für Jahrzehnte war dies wahr. Das britische Kolonialreich erstreckte sich über viele Ecken der Welt. Nach dem Krieg kam es zu einem Prozess der Entkolonialisierung, der Wunsch nach Selbstbestimmung und Eigenverwaltung. \\
Grundsätzlich fing es erst recht mit Indien an. Ihnen die Unabhängigkeit zu gewähren, erfolgte jedoch nur aufgrund einer Unumgänglichkeit. Gandhis erfolgreiche soziale Bewegungen halfen den Blick auf eine Kolonialmacht fundamental zu verändern, und hat vielleicht sogar dazu geführt das gesamte britische Kolonialreich wie man es kannte zum Einsturz zu bringen. \\
In Indien gab es bereits aufgrund der jahrhundertelangen Herrschaft der Briten zahlreiche Aufstände und Konflikte. Doch es sollte bis zum Auftauchen von Mohandas 'Mahatma' Gandhi dauern, bis sich diese intensivierten. Seine Bemühungen ab 1915-1920 zeigten langsam Früchte, und auch der 'durchschnittliche Inder' konnte Stück für Stück überzeugt werden. \clearpage
Gandhi kehrte aus Südafrika zurück, wo er für mehr als 20 Jahre verweilte. Er kehrte zurück als 'Stimme \& Gewissen' tausender rassistisch unterworfener Inder. Im Zuge seiner Rückkehr nach Indien rief er dazu auf, britische Institutionen sowie Produkte gewaltfrei zu boykottieren. Diese Bewegung ist bekannt als 'Swadeshi'. Da seine Bemühungen schlussendlich so berühmt wurden, nannte der erste Premierminister des unabhängigen Indiens 1947, Jawaharlal Nehru, in seiner berühmten Unabhängigkeitsrede Gandhi 'Der Vater unserer Nation, der die Fackel der Freiheit getragen hat und die Dunkelheit beleuchtete, die uns umgab". \\
Während des Zweiten Weltkriegs erhielt Gandhis Bewegung einen enormen Schwung, was für Großbritannien wahrlich zu einer Belastung wurde. Sie wussten von der möglichen Signifikanz der Swadeshi-Bewegung. Zusätzlich gab es Spannungen innerhalb Indiens, welche von zwei Hauptfaktoren bestimmt wurden. Die wirtschaftlichen und menschlichen Ressourcen Großbritanniens waren durch den Kriegseinsatz vollständig ausgelastet. Als zweiter Faktor gilt Japan, welche 1943 in die britische Kolonie Burma eingefallen sind, und weiterhin in Südostasien aggressiv expandiert. \\
Jeder einzelner Faktor war wichtig, um die Briten davon zu überzeugen, dass auf lange Sicht die Macht im Land Indien nicht zu halten ist. Die von Clement Attlee angeführte Labour Party, die bei den Unterhauswahlen 1945 an die Macht gelangt war, nahm sich rasch diesem drängendsten Problems an. \\
Der Indische Nationalkongress und die Muslimliga hatten sich ebenso bereits seit Jahrzehnten für die Unabhängigkeit eingesetzt, waren sich aber über die Umsetzung uneinig. Erstere befürworteten einen gesamtindischen Staat, letztere einen separaten Staat in Gebieten mit muslimischer Mehrheit. Angesichts von Unruhen und eines drohenden Bürgerkriegs erklärte Lord Mountbatten, der letzte britische Vizekönig, das mehrheitlich hinduistische Indien und das mehrheitlich muslimische Pakistan am 15. August 1947 recht überhastet für unabhängig. \\
Die durch Großbritannien festgelegte Grenzziehung machte Dutzende Millionen Menschen zu Angehörigen einer religiösen Minderheit. Die einsetzenden Flüchtlingsströme führten zu Gewalt und Krieg zwischen beiden Gruppen und zu Hunderttausenden von Toten. Burma und Ceylon erlangten ihre Unabhängigkeit 1948. Im Gegensatz zu Indien, Pakistan und Ceylon trat Burma nicht dem Commonwealth of Nations bei. Das Commonwealth of Nations ist heute eine Vereinigung unabhängiger, ex-britischer Staaten, die heute als Nachfolger des British Empire gesehen werden kann. Die Institutionalisierung des British Commonwealth of Nations war Anfang des 20. Jahrhunderts eine Reaktion des Vereinigten Königreiches auf die Autonomiebestrebungen seiner Dominions (Kanada, Südafrika, Australien und Neuseeland) und sollte diese dadurch an das Empire binden.

\subsubsection{Völkerbundmandat Palästina}

Das britische Völkerbundsmandat für Palästina, wo eine arabische Mehrheit mit einer jüdischen Minderheit zusammenlebte, erwies sich für Großbritannien als ähnliches Problem wie Indien. Es wurde zusätzlich verschärft durch die große Anzahl jüdischer Flüchtlinge, die sich nach der Unterdrückung und dem Genozid durch die Nationalsozialisten während des Zweiten Weltkriegs in Palästina niederlassen wollten. Anstatt sich mit der Angelegenheit zu befassen, erklärte die britische Regierung 1947, dass sie im folgenden Jahr ihre Truppen zurückziehen und die Problemlösung den Vereinten Nationen überlassen werde. Sie versuchte dies durch die Ausarbeitung eines Teilungsplans, konnte aber nicht den Palästinakrieg verhindern, der die einseitige Proklamation des Staates Israel zur Folge hatte. \clearpage
In den Fünfzigerjahren erfolgten entscheidende Weichenstellungen für den Abschied vom Empire. Nachdem die Regierung Churchill zu Beginn der Dekade in Sachen Dekolonisation einen eher gemächlichen Takt angeschlagen htte, manövrierte die Suezpolitik den damaligen Premierminister Edens Großbritannien 1956 in erhebliche Turbulenzen. Aber erst die Unruhen in den afrikanischen Kolonien verhalfen einer neuen Dekolonisationsstrategie zum Durchbruch. \\

\subsubsection{Die Suezkrise 1956 und die Folgen}

Die Suezkrise im Jahr 1956 war eine in einen bewaffneten Konflikt mündende Krise zwischen Ägypten (geführt von Präsident Gamal Abdel Nasser) auf der einen und einer Allianz aus Großbritannien, Frankreich und Israel auf der anderen Seite. \\
1956 verstaatlichte Nasser unvermittelt den Sueskanal. Als Reaktion darauf führte der neue Premierminister Anthony Eden Verhandlungen mit den Regierungen Frankreichs und Israels. Ein israelischer Angriff auf Ägypten sollte den Briten und Franzosen als Vorwand dienen, die Sueskanalzone zurückzuerobern. Der US-Präsident Dwight D. Eisenhower war nicht in die Pläne eingeweiht worden und verweigerte aus Verärgerung jegliche Unterstützung. Eisenhower fürchtete auch einen Krieg gegen die Sowjetunion, da Chruschtschow gedroht hatte, den Ägyptern zu Hilfe zu eilen. Die Amerikaner übten Druck aus, indem sie den Verkauf ihrer Pfund-Reserven androhten, was zum Zusammenbruch der britischen Währung geführt hätte. \\
Die europäischen Mächte erhielten in dem Konflikt keinerlei Rückendeckung von Seiten der Vereinigten Staaten. Das Engagement am Kanal, obwohl militärisch erfolgreich, entwickelte sich so gerade für Großbritannien auf Druck der USA zu einer Demütigung. Auf ägyptischer Seite stärkte die Krise trotz militärischer Niederlage massiv die Position Nassers in der arabischen Welt und seinen Panarabismus. \\
Die Suez-Krise hat seither viele Politikanalysten und Historiker beschäftigt. Es liegt nahe, den schmachvollen Rückgang Großbritanniens als engültigen Beleg für dessen imperialen Niedergangs zu interpretieren. \\
Auf internationaler Ebene legte die Krise die stark reduzierten Handlungsspielräume der europäischen Kolonialmächte offen, die durch die Weltöffentlichkeit (UNO) und ihre Abhängigkeit von den USA eingeschränkt waren. Ohne Einwilligung oder gar Unterstützung der Vereinigten Staaten war Großbritannien allein nicht mehr länger handlungsfähig. \\
Diese Erkenntnisse führten auf der imperialen Ebene zu einer Neubewertung der verfügbaren Mittel der Interessendurchsetzung. Ein brachiales und international nicht abgestimmtes imperialistisches Vorgehen galt fortan als unklug.

\subsubsection{Entwicklungen in Afrika}

Kolonialminister Lyttelton meinte 1951: 'Wir alle wollen den Kolonialterritorien dabei helfen, die Selbstregierung innerhalb des britischen Commonwealths zu erreichen [...] Wir sind alle daran interessiert, die ökonomische und soziale Entwicklung voranzutreiben, damit diese mit der politischen Entwicklung Schritt hält.' \\
Bereits in der Zwischenkriegszeit hatten die europäischen Siedler Südrhodesiens darauf gedrängt, ihr Territorium mit Nordrhodesien zu Verschmelzen und mit Njassaland, wo nur wenige Siedler lebten, zumindest in einer Föderation zusammenzuführen. London lehnte dies aber mit Rücksicht auf seine Verpflichtungen der afrikanischen Mehrheitsbevölkerung gegenüber ab. Die Südrhodesier hatten 1923 gegen eine Verschmelzung mit Südafrika votiert und genossen seither de facto Selbstregierung in sämtlichen internen Angelegenheiten. \\
Ghana errang als erste afrikanische Kolonie Großbritanniens 1957 ihre Unabhängigkeit und spielte in Afrika dieselbe Rolle, die Indien für Asien verkörperte - die eines Schrittmachers und Vorbilds für die anderen Territorien unter fremder Herrschaft. \\
Der sparsame Umgang mit dem Begriff Unabhängigkeit soll die Zentrifugalkräfte in den jungen Staaten im Zaum halten: Mit Blick auf die verfassungsrechtliche Entwicklung der Kolonialterritorien sollte der Begriff 'volle Selbstregierung' dem der 'Unabhängigkeit' vorgezogen werden. Letzterer könne zu dem Schluss verleiten, das Ziel der verfassungsrechtlichen Entwicklung sei die Sezession vom Commonwealth. \\
Einen der Grundpfeiler britischer Kolonialpolitik bildete die Unterscheidung zwischen den relativ weit fortgeschrittenen Territorien Westafrikas, wie die Goldküste oder Nigeria, und den Gebieten im Osten des Kontinents, wo die Präsenz der Siedler die Notwendigkeit der Integration von Afrikanern in den Regierungsapparat lange nicht vordringlich erscheinen ließ. \\
Neben der projektierten verfassungsrechtlichen Emanzipation der Kolonien sollte die soziale und wirtschaftliche Entwicklung im Zentrum der britischen Politik stehen und eigentlich die Basis für erstere etablieren. \\
Viele Kritiker sahen sich bald nach Gründung der Föderation in ihren Bedenken hinsichtlich einer drohenden Übervorteilung der Afrikaner bestätigt. Im übrigen hatten die Siedler keinen ausnahmslos guten Leumund in Großbritannien.

\subsubsection{Das Ende des Weltreichs}

Abschließend bleibt zu sagen, dass die beschriebene Dekolonisation zu Beginn der 1980er Jahre bereits weitgehend abgeschlossen war. \\
Die einzige Neuerwerbung war 1955 Rockall gewesen, ein unbewohnter Felsen im Nordatlantik; dadurch sollte die sowjetische Marine daran gehindert werden, Raketentests auf den Hebriden zu beobachten. 1982 besetzte Argentinien die Falklandinseln und berief sich dabei auf Ansprüche aus der spanischen Kolonialzeit. Im anschließenden Falklandkrieg konnten die anfänglich überraschten Briten die Inselgruppe zurückerobern; die Niederlage Argentiniens führte dort zum Sturz der Militärdiktatur. \\
Im selben Jahr wurde Kanada durch das vom britischen Parlament erlassene Kanada-Gesetz 1982 verfassungsrechtlich vollständig vom Mutterland getrennt. Entsprechende Gesetze für Australien und Neuseeland folgten 1986. \\
Im September 1982 verhandelte Premierministerin Margaret Thatcher mit der Regierung der Volksrepublik China über die Zukunft der letzten bedeutenden und bevölkerungsreichsten britischen Kolonie Hongkong. Gemäß den Bestimmungen des Vertrags von Nanking von 1842 hatten die Chinesen Hong Kong Island 'auf ewig' abgetreten. Thatcher wollte an Hongkong festhalten und schlug eine britische Verwaltung unter chinesischer Souveränität vor, was die Chinesen jedoch ablehnten. 1984 vereinbarten beide Regierungen die Einrichtung einer Sonderverwaltungszone, unter dem Prinzip 'Ein Land, zwei Systeme'. Viele Beobachter, darunter der anwesende Prinz Charles, bezeichneten die Übergabezeremonie mit Auslaufen des Pachtvertrags nach 99 Jahren am 30. Juni 1997 als das 'Ende des Empire'.

\clearpage
\bibliographystyle{unsrt}
\bibliography{Pool2_Weinb_5BHIT}

\end{document}
