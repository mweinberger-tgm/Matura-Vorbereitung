\documentclass[letterpaper, 12pt]{article}

%%%%%%%%%%%%%%%%%%%%%%%%%%%%%
% DEFINITIONS
% Change those informations
% If you need umlauts you have to escape them, e.g. for an ü you have to write \"u
\gdef\mytitle{Ausarbeitung}
\gdef\mythema{Pool 6a}

\gdef\mysubject{GGP-Matura}
\gdef\mycourse{5BHIT 2015/16}
\gdef\myauthor{Michael Weinberger}

\gdef\myversion{1.0}
\gdef\mybegin{17. Mai 2016}
\gdef\myfinish{20. Mai 2016}

\gdef\mygrade{}
\gdef\myteacher{Betreuer: Kraus}
%
%%%%%%%%%%%%%%%%%%%%%%%%%%%%%

\input special/preamble.tex

\let\tempsection\section
\renewcommand\section[1]{\vspace{-0.3cm}\tempsection{#1}\vspace{-0.3cm}}
\WithSuffix\newcommand\section*[1]{\tempsection*{#1}}

\let\tempsubsection\subsection
\renewcommand\subsection[1]{\vspace{0cm}\tempsubsection{#1}\vspace{0cm}}

\let\tempsubsubsection\subsubsection
\renewcommand\subsubsection[1]{\vspace{0cm}\tempsubsubsection{#1}\vspace{0cm}}

\linespread{0.94}

\lhead{\mysubject}
\chead{}
\rhead{\bfseries\mythema}
\lfoot{\mycourse}
\cfoot{\thepage}
% Creative Commons license BY
% http://creativecommons.org/licenses/?lang=de
\rfoot{\ccby\hspace{2mm}\myauthor}
\renewcommand{\headrulewidth}{0.4pt}
\renewcommand{\footrulewidth}{0.4pt}

\begin{document}
\parindent 0pt
\parskip 6pt

\pagenumbering{Roman} 
\input{special/title}

\clearpage
\thispagestyle{empty}
\tableofcontents

\newpage
\pagenumbering{arabic}
\pagestyle{fancy}

%\vspace{-0.5cm}

\section{Pool 6a - Wechselwirkungen von Kultur, Gesellschaft und Wirtschaft in der Geschichte \cite{buch}}

\subsection{Flucht, Vertreibung und Deportation - Migration, weltweite Kriege und Zwangsarbeit im 20. Jahrhundert}

Der Erste und Zweite Weltkrieg sowie der folgende globale 'Kalte Krieg' bedeuteten tiefe Einschnitte in die weltpolitische Ordnung und die weltwirschaftlichen Verhältnisse. Europa verlor in kurzer Zeit die über Jahrhunderte errungene Position als globales und politisches Zentrum, die Kolonialreiche waren aus politischen und finanziellen Gründen nicht mehr zu halten. Den 'Kalten Krieg' erlebte das 'Alte Europa' kaum mehr als eigenständig operierender weltpolitischer Akteur. Die weltweiten kriegerischen bzw. kriegsähnlichen Konflikte des 20. Jahrhunderts und deren politische Folgen führten zu einer enormen Zunahme der Zwangswanderungen in Gestalt von Deportation und Zwangsarbeit in den Kriegswirtschaften, Evakuierung und Flucht aus den Kampfzonen ebenso wie Massenausweisung und Vertreibung nach Kriegsende.

\subsubsection{Der Erste Weltkrieg als Motor des Zwangswanderungsgeschehens}

Der Erste Weltkrieg fürhte als 'totaler Krieg' zu einem rapiden Anwachsen der militärischen Kapazitäten der Gegner. Ein Kennzeichen der daraus resultierenden neuen Konfliktdynamik war, dass die militärischen Operationen zum Teil innerhalb weniger Tage und Wochen Millionen von Zivilisten in den Kampfzonen entwurzelten. Allein in den ersten drei Monaten nach dem Angriff Deutschlands flohen beispielsweise 1.4 Millionen Belgier, also ein Fünftel der 1914 knapp 7 Millionen Menschen umfassenden Gesamtbevölkerung des Landes, in die Niederlande, nach Frankreich oder Großbritannien. Weitere Hunderttausende verließen fluchtartig die Kampfzonen in Nord- und Nordostfrankreich, deren Bevölkerung noch ein Jahr nach Kriegsende mit rund 2 Millionen erst wieder rund 40 Prozent des Vorkriegsstandes erreichte. \\
Die Kriegssituation erleichterte bzw. ermöglichte eine staatliche Politik der Zwangsmigration gegenüber missliebigen Minderheiten. Erst der beschleunigte Ausbau der Interventions- und Ordnungskapazitäten der Staaten im Krieg bot die administrativen Instrumente, um Massenausweisungen oder Massenvertreibungen durchzuführen. Darüber hinaus förderte der Erste Weltkrieg die Verbreitung extremer Nationalismen - Fremdenfeindlichkeit wurde lanciert und die Tendenz zur Ausgrenzung von Minderheiten verstärkt. \\
Ein Instrument zum staatlichen Umgang mit 'feindlichen Ausländern' bildete die Internierung. Nicht weniger als 400.000 von ihnen wurden in den kriegsführenden europäischen Staaten 1914-1918 als 'Zivilgefangene' in Lagern festgehalten, Zehntausende darüber hinaus unter Zwang repatriiert. \\
Im Erstn Weltkrieg kam es zudem zur Internationalisierung der Arbeitsmärkte und Heere, die häufig mit Deportation und Zwangsrekrutierung verbunden war: Frankreich und Großbritannien griffen dabei vor allem auf ihre Kolonialbesitzungen und informellen Imperien zurück. Bis Kriegsende rekrutierte Frankreich mehr als 600.000 Soldaten in den Kolonien. Großbritannien mobilisierte dagegen vor allem in Indien, insgesamt verstärkten etwa 1.2 Millionen indische Soldaten weltweit die britischen Truppen. \\
Außerdem lösten die rund 60 Millionen Soldaten aus Europa in den Ländern einen großen Arbeitskräftemängel aus. Gewaltsam festgehaltene Kriegsgefangene arbeiteten sowohl in der Landwirtschaft als auch in der Rüstungsindustrie und im Bergbau, waren in Klein- wie auch in Großbetrieben zu finden und über Hunderttausende von Arbeitsstellen in ganz Europa und in Sibirien verteilt. Das Ende des Ersten Weltkriegs leitete eine Phase millionenfacher Rückwanderungen von Flüchtlingen, Vertriebenen, Evakuierten, Zwangsarbeitskräften und Kriegsgefangenen ein.

\subsubsection{Geschehnisse des Zweiten Weltkrieges}

Ebenso wie der Erste Weltkrieg und dessen unmittelbare Nachkriegszeit (u.a. Weltwirtschaftskrise) wurden auch der zweite globale Konflikt und seine Folgejahre durch Flucht, Vertreibung, Deportation und Zwangsarbeit geprägt, allerdings in noch erheblich größeren Dimensionen. Die Bevölkerungsverluste waren wesentlich höher: Wahrscheinlich hat der Zweite Weltkrieg 55 bis 60 Millionen Menschen das Leben gekostet. Anders als im Ersten Weltkrieg lag dabei die Zahl der Getöteten unter der Zivilbevölkerung höher als unter den Soldaten. In Europa kann die Zahl der Flüchtlinge allein in der militärischen Expansionsphase des nationalsozialistischen Deutschland zwischen 1939 und 1945 zu beobachtenden Massenzwangsauswanderungen, so kann für den Zweiten Weltkrieg insgesamt von 50 bis 60 Millionen Flüchtlingen, Vertriebenen und Deportierten ausgegangen werden. Das waren mehr als 10 Prozent aller Menschen in Europa. Auch der Krieg im pazifischen Raum ließ die Zahl der Vertriebenen rasch ansteigen - und zwar schon bevor in Europa die Kämpfe begonnen hatten. \\
Das nationalsozialistische 'Dritte Reich' war nur deshalb in der Lage, den Zweiten Weltkrieg beinahe sechs Jahre lang zu führen, weil es ihn als Beutekrieg geplant hatte. Die mit Deutschland verbündeten Staaten sowie die von 1938 an erworbenen bzw. eroberten Länder und Landesteile hatten dabei die Aufgabe, mit Produktionskapazitäten, Rohstoffen und mit ihrer Bevölkerung der deutschen Kriegswirtschaft zu dienen. Im Laufe des Krieges stieg die Bedeutung der geraubten Güter und Menschen für die deutsche Kriegswirtschaft immens an: Im Oktober 1944 wurden fast 8 Millionen ausländische Zwangsarbeitskräfte in Deutschland gezählt, darunter knapp 6 Millionen Zivilisten und rund 2 Millionen Kriegsgefangene. Sie stammten aus insgesamt 26 verschiedenen Ländern. Die UdSSR dominierte als Herkunftsland der Zwangsarbeitskräfte mit einem Anteil von mehr als einem Drittel (2.8 Millionen) an ihrer Gesamtzahl, 1.7 Millionen kamen aus Polen und 1.2 Millionen aus Frankreich, jeweils mehrere Hunderttausend aus Italien, den Niederlanden, Belgien, der Tschechoslowakei und Jugoslawien. Das enorme wirtschaftliche Gewicht der ausländischen Zwangsarbeitskräfte zeigt sich im Anteil an der Gesamtbeschäftigung: Insgesamt stellten sie im September 1944 etwa ein Drittel der Beschäftigten. Wiederum ein Drittel der ausländischen Arbeitskräfte waren Frauen - ein Großteil jünger als 20 Jahre. Insgesamt gesehen lag das Durchschnittsalter bei 20 bis 24 Jahren. \\
Deutschland wurde mit einem System von über 20.000 Lagern für ausländische Zwangsarbeitskräfte überzogen. Ausländische Arbeitskräfte gab es überall, in der Stadt wie auf dem Land. Entsprechend der rassistischen nationalsozialistischen Weltanschauung behandelten die deutschen Behörden die ausländischen Zwangsarbeiter je nach Nationalität ganz unterschiedlich. Jene aus verbündeten Ländern sowie aus besetzten Gebieten im Westen waren in den Arbeits- und Lebensverhältnissen dabei weitaus besser gestellt als jene aus dem Osten, jene aus den besetzten Gebieten der UdSSR - neben den Häftlingen in der KZ-Rüstungsproduktion im Reichsgebiet - den schlechtesten Arbeits- und Lebensbedingungen unterworfen.\\
Das Interesse der deutschen Eroberer ging in den besetzten Gebieten vor allem Ost- und Ostmitteleuropas über die wirtschaftliche Ausbeutung deutlich hinaus, denn die Besatzungspolitik zielte auf die Etablierung einer streng nach rassistischen Kriterien ausgerichteten deutschen Ordnung, deren wesentliche Elemente Planung und weitreichende Umsetzung von Umsiedlungen sowie Vertreibungen und Deportationen ganzer Bevölkerungen zugunsten eines vorgeblichen deutschen 'Volkes ohne Raum' waren. Etwa 9 Millionen Menschen waren davon betroffen. Zwischen 1939 und 1944 wurden als Nutznießer der Umsiedlungen eine Million Menschen deutscher Herkunft aus ihren außerhalb der Reichsgrenzen gelegenen Siedlungsgebieten in Süd-, Südost-, Ostmittel- und Osteuropa 'heim ins Reich' geholt, vor allem, um sie in den in Polen und der Tschechoslowakei eroberten, dem Reich unmittelbar angegliederten Gebieten anzusiedeln. Voraussetzung für die Ansiedlung dieser 'Volksdeutschen' war immer die Deportation der ansässigen polnischen, tschechischen und jüdischen bevölkerung, die 1939/40 im großen Maßstab eingeleitet worden war und im Völkermord endete. 1940/41 etwa wurden ca. 1.2 Millionen Polen und Juden aus den 'Reichsgauen' im Gebiet des heutigen Polens vertrieben zugunsten der neu anzusiedelnden 'Volksdeutschen'. Das sollte aber nur Anfang sein, die Gesamtplanung für dieses Gebiet lag bereits vor, denn von den mehr als 10 Millionen Menschen, die in diesem Gebiet lebten, galten nur 1.7 Millionen als 'eindeutschfähig', 7.8 Millionen Polen und 700.000 Juden sollten vertrieben werden.

\subsubsection{Kriegsfolgewanderungen}

Die letzten Umsiedlungen 'heim ins Reich' von 250.000 'Volksdeutschen' aus Wolhynien, Galizien und Siebenbürgen 1944 hatten schon deutlich den Charakter einer Fluchtbewegung vor der Roten Armee, die im August 1944 in Ostpreußen die Grenze des Deutschen Reiches erreichte und sie im Oktober des Jahres erstmals überschritt. In den Ostprovinzen des Reiches und in den deutschen Siedlungsgebieten jenseits der Grenzen in Ost-, Ostmittel- und Südosteuropa lebten rund 18 Millionen Reichsdeutsche und 'Volksdeutsche'. Etwa 14 Millionen von ihnen, der weitaus überwiegende Teil also, flüchteten in der Endphase des Krieges in Richtung Westen oder wurden nach Kriegsende vertrieben bzw. deportiert. Die Bilanz zeigen die Zahlen der Volkszählung von 1950: Knapp 12.5 Millionen Flüchtlinge und Vertriebene waren aus den nunmehr in polnischen und sowjetischen Besitz übergegangenen ehemaligen Ostgebieten des Deutschen Reiches sowie aus den Siedlungsgebieten der 'Volksdeutschen' in die Bundesrepublik Deutschland und in die DDR gelangt. Weitere 500.000 lebten in Österreich und anderen Ländern. \\
Die Flüchtlinge und Vertriebenen aber bildeten im Deutschland der unmittelbaren Nachkriegszeit nicht die einzige große Gruppe von Zwangswanderern. Hinzu kamen die 11 Millionen 'Displaced Persons' (kurz DPs), ehemalige ausländische Zwangsarbeitskräfte, deren Rück- und Weitertransport Monate und Jahre in Anspruch nahm. In den vier Besatzungszonen gab es nach Kriegsende zudem noch 10 Millionen Menschen, die vor den Bombenangriffen geflohen waren oder evakuiert wurden und nicht selten erst nach Jahren in ihre Heimatorte zurückkehren konnten. Innerhalb eines Jahres nach Kriegsende 1945 wurden zudem rund 5 der insgesamt 9 Millionen deutschen Kriegsgefangenen aus den Internierungslagern entlassen. 20 verschiedene Staaten hatten deutsche Kriegsgefangene in ihrem Gewahrsam. \\
Der Zweite Weltkrieg hatte die Lebensgrundlagen von Millionen Menschen zerstört; das Verlassen des Kontinents erschien vielen als ein Weg aus der Trümmerlandschaft. Dennoch lag die transkontinentale Abwanderung im Kriegs- und Nachkriegsjahrzent 1941-1950 mit 2.3 Millionen niedrig. Während des Krieges gab es faktisch keine Überseemigration, nach dem Krieg lief sie zunächst sehr langsam an und unterschied sich wesentlich von jener des 19. und frühen 20. Jahrhunderts. Von dem Migranten selbst organisierte Reisen gab es kaum noch, einen wesentlichen Anteil hatte etwa die von internationalen Hilfsorganisationen organisierte Abwanderung der 'Displaced Persons' aus Europa. Mit Hilfe der 'International Refugee Organization' und über ein international abgestimmtes Aufnahmeprogramm konnten zwischen 1947 und 1951 mehr als 700.000 DPs Westdeutschland verlassen. \\
Die Aufnahme dieser DP-Programme war ein Ergebnis des 'Kalten Krieges', der von den späten 1940er bis in die späten 1980er Jahre die globale Politik prägte. Anfangs waren die Westalliierten noch der Aufforderung der UdSSR nachgekommenn, DPs auch unter Zwang in die UdSSR zurückzusenden. Weil sich aber die politischen Differenzen zwischen Ost und West immer weiter verschärften, entwickelten die Westmächte eigene Strategien zum Umgang mit jenen DPs, die nicht in ihre ost- und mitteleuropäischen Herkunftsländer zurückkehren wollten.

\subsubsection{Migration und 'Kalter Krieg'}

Trotz der Verwendung des Terminus 'Krieg' verweist der 'Kalte Krieg' nicht auf direkte (zwischen den beiden verfeindeten 'Supermächten' UdSSR und USA ausgetragene) militärische Konflikte, sondern meint vielmehr eine Phase permanenten 'Nicht-Friedens', einen von beiden Seiten aktiv betriebenen kriegsähnlichen Zustand. Ein zentrales Element des Systemkonflikts bildete der jeweils vertretene unvereinbare politisch-weltanschauliche Absolutheits- bzw. Universalanspruch. Als langwährender Rüstungsverlauf mit teuren Waffentechnologien war der 'Kalte Krieg' eine Auseinandersetzung, die einen erheblichen Teil der finanziellen und ökonomischen Ressourcen im Osten wie im Westen band. \\
Für die globale Migrationssituation war die (ideologische) Teilung der Welt von hohem Gewicht. Die UdSSR hatte bereits in der Zwischenkriegszeit ein an den Erfordernissen einer gewaltsamen Industrialisierungspolitik orientiertes Migrationsregime entwickelt, das auf die restriktive Lenkung von Arbeitskräften im Innern und auf Beschränkung der Abwanderung ausgerichtet war. Nach dem Ende des Zweiten Weltkriegs gingen die neuen Satellitenstaaten der UdSSR den sowjetischen Weg. Migratorisch wurde die Welt in zwei Blöcke geteilt, Arbeitsmigration fand zwischen Ost und West nicht mehr statt. Die Bewegungen beschränkten sich meist auf Flucht oder Ausweisung von Dissidenten aus dem Osten in den Westen oder auf Phasen, in denen die Destabilisierung eines Staatswesens im Osten den kurzzeitigen Zusammenbruch der restriktiven Grenzregime zur Folge hatte. Das galt vor allem für die Aufstände in Ungarn 1956 und in der Tschechoslowakei 1968, deren Niederschlagung jeweils zur Abwanderung Hunderttausender führte. Einen Sonderfall bildete bis zum Bau der Berliner Mauer 1961 die DDR. Zwar wurde die innerdeutsche Grenze bereits Anfang der 1950er Jahre unüberwindbar armiert, die besondere Stellung Berlins aber ließ Grenzsicherungsmaßnahmen zwischen den alliierten Sektoren der ehemaligen Reichshauptstadt lange nicht zu, sodass DDR und UdSSR die Abwanderung kaum kontrollieren konnten: Wahrscheinlich wanderten von der Gründung der beiden deutschen Staaten 1949 bis zum Bau der Mauer 1961 über 3 Millionen Menschen aus der DDR in die Bundesrepublik. Andere migratorische Wirkungen des 'Kalten Krieges' betrafen jene Weltregionen, in denen der Konflikt als 'Stellvertreterkrieg' ausgetragen wurde: Vor allem die Kriege in Korea 1950— 1953, in Vietnam 1961-1975 und in Afghanistan 1979-1989, an denen jeweils eine der beiden Weltmächte in großem Maßstab militärisch beteiligt war, während der globale Gegner durch die Lieferung von Rüstungsgütern sowie durch finanzielle, materielle und ideelle Hilfen den jeweiligen Kriegsgegner unterstützte, bedingten große Flucht- und Vertreibungsbewegungen. Meist führten sie nicht über die Grenzen der betroffenen Staaten hinaus oder betrafen höchstens Grenzregionen benachbarter Staaten. Die Zahl der Zwangsmigranten war vor allem in Vietnam auch deshalb sehr hoch, weil die US-Truppen Umsiedlungen zu einem Element der Kriegführung bzw. zu einem Element der 'Befriedung' guerillagefährdeter oder eroberter Gebiete machten. Sie griffen dabei auf Erfahrungen aus anderen Dekolonisationskonflikten zurück: Bereits im Krieg der britischen Kolonialmacht in Malaya gegen eine kommunistische Guerilla 1958 bis 1960 war die Umsiedlung eines großen Teils der Minderheit der Chinesen als zentrales Element einer erfolgreichen Aufstands-bekämpfung verstanden worden. Die im Sinne dieser Strategie 
• 1[7° 1 i • 
kornmu sehen ti mische rung b( passier China. names( Hilfe v Todesr den v Aufna langte trauen 

\clearpage
\bibliographystyle{unsrt}
\bibliography{Pool6_Weinb_5BHIT}

\end{document}
