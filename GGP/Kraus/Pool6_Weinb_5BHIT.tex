\documentclass[letterpaper, 12pt]{article}

%%%%%%%%%%%%%%%%%%%%%%%%%%%%%
% DEFINITIONS
% Change those informations
% If you need umlauts you have to escape them, e.g. for an ü you have to write \"u
\gdef\mytitle{Ausarbeitung}
\gdef\mythema{Pool 6a}

\gdef\mysubject{GGP-Matura}
\gdef\mycourse{5BHIT 2015/16}
\gdef\myauthor{Michael Weinberger}

\gdef\myversion{1.0}
\gdef\mybegin{17. Mai 2016}
\gdef\myfinish{20. Mai 2016}

\gdef\mygrade{}
\gdef\myteacher{Betreuer: Kraus}
%
%%%%%%%%%%%%%%%%%%%%%%%%%%%%%

\input special/preamble.tex

\let\tempsection\section
\renewcommand\section[1]{\vspace{-0.3cm}\tempsection{#1}\vspace{-0.3cm}}
\WithSuffix\newcommand\section*[1]{\tempsection*{#1}}

\let\tempsubsection\subsection
\renewcommand\subsection[1]{\vspace{0cm}\tempsubsection{#1}\vspace{0cm}}

\let\tempsubsubsection\subsubsection
\renewcommand\subsubsection[1]{\vspace{0cm}\tempsubsubsection{#1}\vspace{0cm}}

\linespread{0.94}

\lhead{\mysubject}
\chead{}
\rhead{\bfseries\mythema}
\lfoot{\mycourse}
\cfoot{\thepage}
% Creative Commons license BY
% http://creativecommons.org/licenses/?lang=de
\rfoot{\ccby\hspace{2mm}\myauthor}
\renewcommand{\headrulewidth}{0.4pt}
\renewcommand{\footrulewidth}{0.4pt}

\begin{document}
\parindent 0pt
\parskip 6pt

\pagenumbering{Roman} 
\input{special/title}

\clearpage
\thispagestyle{empty}
\tableofcontents

\newpage
\pagenumbering{arabic}
\pagestyle{fancy}

%\vspace{-0.5cm}

\section{Pool 6a - Wechselwirkungen von Kultur, Gesellschaft und Wirtschaft in der Geschichte \cite{buch}}

\subsection{Flucht, Vertreibung und Deportation - Migration, weltweite Kriege und Zwangsarbeit im 20. Jahrhundert}

Der Erste und Zweite Weltkrieg sowie der folgende globale 'Kalte Krieg' bedeuteten tiefe Einschnitte in die weltpolitische Ordnung und die weltwirschaftlichen Verhältnisse. Europa verlor in kurzer Zeit die über Jahrhunderte errungene Position als globales und politisches Zentrum, die Kolonialreiche waren aus politischen und finanziellen Gründen nicht mehr zu halten. Den 'Kalten Krieg' erlebte das 'Alte Europa' kaum mehr als eigenständig operierender weltpolitischer Akteur. Die weltweiten kriegerischen bzw. kriegsähnlichen Konflikte des 20. Jahrhunderts und deren politische Folgen führten zu einer enormen Zunahme der Zwangswanderungen in Gestalt von Deportation und Zwangsarbeit in den Kriegswirtschaften, Evakuierung und Flucht aus den Kampfzonen ebenso wie Massenausweisung und Vertreibung nach Kriegsende.

\subsubsection{Der Erste Weltkrieg als Motor des Zwangswanderungsgeschehens}

Der Erste Weltkrieg fürhte als 'totaler Krieg' zu einem rapiden Anwachsen der militärischen Kapazitäten der Gegner. Ein Kennzeichen der daraus resultierenden neuen Konfliktdynamik war, dass die militärischen Operationen zum Teil innerhalb weniger Tage und Wochen Millionen von Zivilisten in den Kampfzonen entwurzelten. Allein in den ersten drei Monaten nach dem Angriff Deutschlands flohen beispielsweise 1.4 Millionen Belgier, also ein Fünftel der 1914 knapp 7 Millionen Menschen umfassenden Gesamtbevölkerung des Landes, in die Niederlande, nach Frankreich oder Großbritannien. Weitere Hunderttausende verließen fluchtartig die Kampfzonen in Nord- und Nordostfrankreich, deren Bevölkerung noch ein Jahr nach Kriegsende mit rund 2 Millionen erst wieder rund 40 Prozent des Vorkriegsstandes erreichte. \\
Die Kriegssituation erleichterte bzw. ermöglichte eine staatliche Politik der Zwangsmigration gegenüber missliebigen Minderheiten. Erst der beschleunigte Ausbau der Interventions- und Ordnungskapazitäten der Staaten im Krieg bot die administrativen Instrumente, um Massenausweisungen oder Massenvertreibungen durchzuführen. Darüber hinaus förderte der Erste Weltkrieg die Verbreitung extremer Nationalismen - Fremdenfeindlichkeit wurde lanciert und die Tendenz zur Ausgrenzung von Minderheiten verstärkt. \\
Ein Instrument zum staatlichen Umgang mit 'feindlichen Ausländern' bildete die Internierung. Nicht weniger als 400.000 von ihnen wurden in den kriegsführenden europäischen Staaten 1914-1918 als 'Zivilgefangene' in Lagern festgehalten, Zehntausende darüber hinaus unter Zwang repatriiert. \\
Im Erstn Weltkrieg kam es zudem zur Internationalisierung der Arbeitsmärkte und Heere, die häufig mit Deportation und Zwangsrekrutierung verbunden war: Frankreich und Großbritannien griffen dabei vor allem auf ihre Kolonialbesitzungen und informellen Imperien zurück. Bis Kriegsende rekrutierte Frankreich mehr als 600.000 Soldaten in den Kolonien. Großbritannien mobilisierte dagegen vor allem in Indien, insgesamt verstärkten etwa 1.2 Millionen indische Soldaten weltweit die britischen Truppen. \\
Außerdem lösten die rund 60 Millionen Soldaten aus Europa in den Ländern einen großen Arbeitskräftemängel aus. Gewaltsam festgehaltene Kriegsgefangene arbeiteten sowohl in der Landwirtschaft als auch in der Rüstungsindustrie und im Bergbau, waren in Klein- wie auch in Großbetrieben zu finden und über Hunderttausende von Arbeitsstellen in ganz Europa und in Sibirien verteilt. Das Ende des Ersten Weltkriegs leitete eine Phase millionenfacher Rückwanderungen von Flüchtlingen, Vertriebenen, Evakuierten, Zwangsarbeitskräften und Kriegsgefangenen ein.

\subsubsection{Geschehnisse des Zweiten Weltkrieges}

Ebenso wie der Erste Weltkrieg und dessen unmittelbare Nachkriegszeit (u.a. Weltwirtschaftskrise) wurden auch der zweite globale Konflikt und seine Folgejahre durch Flucht, Vertreibung, Deportation und Zwangsarbeit geprägt, allerdings in noch erheblich größeren Dimensionen. Die Bevölkerungsverluste waren wesentlich höher: Wahrscheinlich hat der Zweite Weltkrieg 55 bis 60 Millionen Menschen das Leben gekostet. Anders als im Ersten Weltkrieg lag dabei die Zahl der Getöteten unter der Zivilbevölkerung höher als unter den Soldaten. In Europa kann die Zahl der Flüchtlinge allein in der militärischen Expansionsphase des nationalsozialistischen Deutschland zwischen 1939 und 1945 zu beobachtenden Massenzwangsauswanderungen, so kann für den Zweiten Weltkrieg insgesamt von 50 bis 60 Millionen Flüchtlingen, Vertriebenen und Deportierten ausgegangen werden. Das waren mehr als 10 Prozent aller Menschen in Europa. Auch der Krieg im pazifischen Raum ließ die Zahl der Vertriebenen rasch ansteigen - und zwar schon bevor in Europa die Kämpfe begonnen hatten. \\
Das nationalsozialistische 'Dritte Reich' war nur deshalb in der Lage, den Zweiten Weltkrieg beinahe sechs Jahre lang zu führen, weil es ihn als Beutekrieg geplant hatte. Die mit Deutschland verbündeten Staaten sowie die von 1938 an erworbenen bzw. eroberten Länder und Landesteile hatten dabei die Aufgabe, mit Produktionskapazitäten, Rohstoffen und mit ihrer Bevölkerung der deutschen Kriegswirtschaft zu dienen. Im Laufe des Krieges stieg die Bedeutung der geraubten Güter und Menschen für die deutsche Kriegswirtschaft immens an: Im Oktober 1944 wurden fast 8 Millionen ausländische Zwangsarbeitskräfte in Deutschland gezählt, darunter knapp 6 Millionen Zivilisten und rund 2 Millionen Kriegsgefangene. Sie stammten 

\clearpage
\bibliographystyle{unsrt}
\bibliography{Pool6_Weinb_5BHIT}

\end{document}
