\documentclass[letterpaper, 12pt]{article}

%%%%%%%%%%%%%%%%%%%%%%%%%%%%%
% DEFINITIONS
% Change those informations
% If you need umlauts you have to escape them, e.g. for an ü you have to write \"u
\gdef\mytitle{Ausarbeitung}
\gdef\mythema{Pool 8b}

\gdef\mysubject{GGP-Matura}
\gdef\mycourse{5BHIT 2015/16}
\gdef\myauthor{Michael Weinberger}

\gdef\myversion{1.0}
\gdef\mybegin{17. Mai 2016}
\gdef\myfinish{20. Mai 2016}

\gdef\mygrade{}
\gdef\myteacher{Betreuer: Kraus}
%
%%%%%%%%%%%%%%%%%%%%%%%%%%%%%

\input special/preamble.tex

\let\tempsection\section
\renewcommand\section[1]{\vspace{-0.3cm}\tempsection{#1}\vspace{-0.3cm}}
\WithSuffix\newcommand\section*[1]{\tempsection*{#1}}

\let\tempsubsection\subsection
\renewcommand\subsection[1]{\vspace{0cm}\tempsubsection{#1}\vspace{0cm}}

\let\tempsubsubsection\subsubsection
\renewcommand\subsubsection[1]{\vspace{0cm}\tempsubsubsection{#1}\vspace{0cm}}

\linespread{0.94}

\lhead{\mysubject}
\chead{}
\rhead{\bfseries\mythema}
\lfoot{\mycourse}
\cfoot{\thepage}
% Creative Commons license BY
% http://creativecommons.org/licenses/?lang=de
\rfoot{\ccby\hspace{2mm}\myauthor}
\renewcommand{\headrulewidth}{0.4pt}
\renewcommand{\footrulewidth}{0.4pt}

\begin{document}
\parindent 0pt
\parskip 6pt

\pagenumbering{Roman} 
\input{special/title}

\clearpage
\thispagestyle{empty}
\tableofcontents

\newpage
\pagenumbering{arabic}
\pagestyle{fancy}

%\vspace{-0.5cm}

\section{Pool 8b - Politische Ideologien, Systeme und Akteure \cite{buch}}

\subsection{Nationalsozialismus und Verbündete, Gedankengut, Historie}

\begin{abstract}

Nationalsozialismus bezeichnet eine politische Bewegung, die in Deutschland in den Krisen nach dem Ersten Weltkrieg entstand, 1933 die Weimarer Demokratie beendete und eine Diktatur (das sogenannte Dritte Reich) errichtete. Der Nationalsozialismus verfolgte extrem nationalistische, antisemitische, rassistische und imperialistische Ziele, die bereits in Hitlers Buch 'Mein Kampf' (erschienen 1925) niedergelegt worden waren. Politisch schloss der Nationalsozialismus an die radikale Kritik und Ablehnung der demokratischen Prinzipien an (die auch in konservativen Kreisen üblich waren) und bekämpfte den Friedensvertrag von Versailles. Der Nationalsozialismus war keine geschlossene Lehre, sondern begründete eine 'Weltanschauung', in deren Mittelpunkt die Idee des 'arischen Herrenvolkes' stand, das sich aller Mittel zu bedienen hat, um sich 'Lebensraum' zu schaffen, andere (angeblich minderwertige) Völker und Nationen zu unterdrücken und die Welt vom (angeblich einzig Schuldigen, dem) Judentum zu befreien. Zum 'Rasse'- und 'Lebensraum'-Gedanken trat als drittes Element ein fanatischer Antibolschewismus. Die Verachtung des Menschen im Nationalsozialismus fand Ausdruck in der fabrikmäßigen Tötung von Millionen wehrloser Opfer (vor allem Juden, 'Fremdvölkische', aber auch 'Asoziale'/Andersdenkende u. a.) in den Konzentrationslagern und in einem bis dahin unbekannten Vernichtungsfeldzug gegen die europäischen Nachbarn. Die nationalsozialistische Diktatur etablierte ein Herrschaftssystem, in dem sich autoritäres Führerprinzip (Befehl und Unterwerfung), hemmungsloser Aktionismus, ein ungeregeltes Nebeneinander von Staat und Partei (NSDAP), planvolle Kriegswirtschaft und 'perfekte Improvisationen' miteinander verbanden und durch eine Kombination von Überzeugung und Unterdrückung, Mitläufertum und Terror zusammengehalten wurden. Politisches Resultat der Diktatur des Nationalsozialismus war die völlige Neuordnung der Gewichte zwischen den Staaten Europas (und der Welt) und die Verkleinerung und die Teilung Deutschlands.

\end{abstract}

\subsubsection{Ursachen für den Aufstieg der NSDAP}

"Das Jahr 1932 hat Hitlers Glück und Ende gebracht. Am 31. Juli hatte sein Aufstieg den Höhepunkt erreicht, am 13. August begann der Niedergang, als der Reichspräsident den Stuhl, den er ihm nicht zum Sitzen anbot, vor die Tür stellte. Seitdem ist das Hitlertum in einem Zusammenbruch, dessen Ausmaß und Tempo dem seines eigenen Aufstiegs vergleichbar ist. Das Hitlertum stirbt an seinem eigenen Lebensgesetz." (liberaler Publizist, 1932)\\
Das war eine durchaus optimistische Einschätzung. Auch wenn sie sich einige Wochen später als dramatische Fehlkalkulationen erweisen sollte. \\
Nicht viel später hatte Hitler alles in der Hand. Die Polizeikräfte, eine unerbittliche Zensur, die die Presse vollständig zähmte. Hitler beherrscht die einzelnen deutschen Länder durch die Statthalter, die er an ihre Spitze gestellt hat. Die Städte werden von jetzt an verwaltet durch Bürgermeister und Stadträte aus seiner Anhängerschaft. Die Regierungen der Länder und die Landtage sind in den Händen seiner Parteigänger. Alle öffentlichen Verwaltungen wurden gesäubert. Die politischen Parteien sind verschwunden. \clearpage
Heute wie damals drängt sich die Frage auf, wie in einer so kurzen Zeit ein etabliertes und differenziertes System von politischen Parteien und gesellschaftlichen Verbänden, von Parlamenten und Verwaltungen zusammenbrechen oder sich selbst aufgeben konnte. Auch fragt sich, wie der rasante und scheinbar unaufhaltsame Aufstieg eines politischen Agitators zu erklären ist, der bis zu seinem 30. Lebensjahr ein politisch und sozialer Niemand war und der in den verbleibenden 26 Lebensjahren die Geschichte zutiefst geprägt hat. Diese Zeitspanne wurde geprägt von einem deutschen Diktator, der fast bis zu seinem Ende auf eine gläubige Gefolgschaft und Zustimmungsbereitschaft der großen Mehrheit der Deutschen setzen konnte, der einen Völkermord und einen Krieg anstiftete und damit einen der größten Zivilisationsbrüche der Neuzeit verursachte. Wie konnte er mit seiner Massenbewegung einen hoch entwickelten und modernen Industriestaat mit einer großen kulturellen Tradition unter seine diktatorische Gewalt bringen? Wie war es möglich, dass die überwiegende Mehrheit der Deutschen sich mit diesem Unrechtsregime arrangiert hat? \\

Einigkeit besteht in der historischen Forschung darin, dass es keine einfachen Erklärungen für Aufstieg und Fall des Nationalsozialismus, für die Verlockungen und die Gewalt im Führerstaat gibt. So kann weder die nationalsozialistische Ideologie und Propaganda allein die Massenwirksamkeit des Nationalsozialismus erklären, denn dort wurde nur verkündet, was man auch anderswo hören konnte; noch kann es die vermeintliche politische Genialität oder Suggestivkraft Hitlers, denn selbst wenn diese von der Parteipropaganda unaufhörlich herausgestellt wurde, bedurfte es erst einer entsprechenden Erwartungshaltung beim Publikum, um eine politische Wirkung zu erzielen. Auch der Terror der Sturmabteilung (SA) kann den Aufstieg des Nationalsozialismus allein nicht erklären. Ebenso wenig die politischen und sozialen Umstände, die immer wieder genannt werden: der Versailler Vertrag (1919) und die kommunistische Revolutionsdrohung aus Moskau, die Massenarbeitslosigkeit oder die sozioökonomischen Interessen der Großindustrie und des Großgrundbesitzes. Keiner dieser Faktoren kann bei einer historischen Erklärung übersehen werden, aber für sich allein reicht weder der eine noch der andere für die Erklärung des nationalsozialistischen Aufstiegs zur Macht noch der Politik des Führerstaates aus. Sie verschränkten sich vielmehr wechselseitig. In einem doppelgleisigen Prozess des Machtverfalls bzw. -verlustes der Demokratie einerseits und der politisch-sozialen Expansion der nationalsozialistischen Bewegung andererseits wurde der politische Handlungsspielraum zuerst der demokratischen, dann aber auch der konservativ-autoritären Kräfte zunehmend eingeengt. Dieser Prozess wurde beschleunigt durch politische Fehleinschätzungen, persönliche Machtkämpfe und Intrigen. 

\subsubsection{Das Gedankengut der Nationalsozialisten}

Die Folgen des Zweiten Weltkrieges sind in seinen politischen, wirtschaftlichen, sozialen und kulturellen Bereichen bis zum heutigen Tag spürbar. Hauptverantwortlicher für einen Krieg, der über 50 Millionen Menschenleben gekostet hat, waren die Nationalsozialisten. An Ihrer Spitze: Adolf Hitler. Der Nationalsozialismus war nach innen totalitär, erkannte keine Grundrechte des Einzelnen an und bekämpfte Kommunismus, Sozialismus, Liberalismus, Menschen mosaischen Glaubens und auch die christlichen Kirchen. Nach außen war er aggressiv, expansiv mit nihilistischer Prägung.

\ceparagraph{Die Ideologie}

Eine Ideologie ist eine weltanschauliche Konzeption (Anschauung, Leitidee), die hauptsächlich der Erreichung politischer und wirtschaftlicher Ziele dienen. \\
Durch völkisch-antisemitische, nationalistische und imperialistische, rassenbiologische und sozialdarwinistische, antidemokratische und antimarxistische Vorstellungen wurde Hitlers persönliches Weltbild geprägt, das von zwei Ideensträngen zusammen gehalten wurde. Von einem radikalen, universalen Rassensemitismus und von der Lebensraumtheorie. Als sich Hitler in Gefangenschaft befand, schrieb er sein Werk: "Mein Kampf", in das er seine Weltanschauungen zusammenfasste. \\

\ceparagraph{Innenpolitische Grundsätze der NSDAP}

\ceparagraph{Absichten \& Ziele der nationalsozialistischen Herrschaft}

\ceparagraph{Die Uniformierung des Lebens}

\ceparagraph{Wichtigste Ideologien der NSDAP} 

\ceparagraph{Weltanschauung} 

\subsubsection{Die Nürnberger Gesetze}

\subsubsection{Verbündete \& Ähnlichkeiten}

\clearpage
\bibliographystyle{unsrt}
\bibliography{Pool8_Weinb_5BHIT}

\end{document}
