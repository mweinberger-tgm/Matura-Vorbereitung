\documentclass[letterpaper, 12pt]{article}

%%%%%%%%%%%%%%%%%%%%%%%%%%%%%
% DEFINITIONS
% Change those informations
% If you need umlauts you have to escape them, e.g. for an ü you have to write \"u
\gdef\mytitle{Ausarbeitung}
\gdef\mythema{Pool 1a}

\gdef\mysubject{GGP-Matura}
\gdef\mycourse{5BHIT 2015/16}
\gdef\myauthor{Michael Weinberger}

\gdef\myversion{1.0}
\gdef\mybegin{17. Mai 2016}
\gdef\myfinish{20. Mai 2016}

\gdef\mygrade{}
\gdef\myteacher{Betreuer: Kraus}
%
%%%%%%%%%%%%%%%%%%%%%%%%%%%%%

\input special/preamble.tex

\let\tempsection\section
\renewcommand\section[1]{\vspace{-0.3cm}\tempsection{#1}\vspace{-0.3cm}}
\WithSuffix\newcommand\section*[1]{\tempsection*{#1}}

\let\tempsubsection\subsection
\renewcommand\subsection[1]{\vspace{0cm}\tempsubsection{#1}\vspace{0cm}}

\let\tempsubsubsection\subsubsection
\renewcommand\subsubsection[1]{\vspace{0cm}\tempsubsubsection{#1}\vspace{0cm}}

\linespread{0.94}

\lhead{\mysubject}
\chead{}
\rhead{\bfseries\mythema}
\lfoot{\mycourse}
\cfoot{\thepage}
% Creative Commons license BY
% http://creativecommons.org/licenses/?lang=de
\rfoot{\ccby\hspace{2mm}\myauthor}
\renewcommand{\headrulewidth}{0.4pt}
\renewcommand{\footrulewidth}{0.4pt}

\begin{document}
\parindent 0pt
\parskip 6pt

\pagenumbering{Roman} 
\input{special/title}

\clearpage
\thispagestyle{empty}
\tableofcontents

\newpage
\pagenumbering{arabic}
\pagestyle{fancy}

%\vspace{-0.5cm}

\section{Pool 1a - Europa im Wandel \cite{buch}}

\subsection{Entstehung und Entwicklung der EU}

\subsubsection{Der Europagedanke in der Nachkriegszeit}

Ein Jahr vor
Kriegsende, im Frühjahr und Sommer 1944, arbeiteten Widerstandsgruppen aus neun europäischen Staaten, darunter
Deutschland, eine gemeinsame Erklärung aus. Sie beriefen sieh
auf die Ziele, die 1941 in der Atlantik—Charta vom britischen
Premierminister Winston Churchill und dem amerikanischen
Präsidenten Franklin D. Roosevelt wurden waren: Demokratie, soziale Gerechtigkeit und Achtung der Menschenrechte. Diese Ziele aber könnten nur erreicht werden, wenn die verschiedenen Länder der Welt sich bereit erklären, das Dogma der absoluten Staatssouveränität abzustreifen, wie sie definierten. \\
Da der Schlüssel zum Frieden in Europa liege, sei zunächst eine mit starken Institutionen ausgestattete europäische 'Bundesordnung' vonnöten, an die sämtliche außen- und sicherheitspolitischen Kompetenzen abzutreten seien. Des Weiteren spielt das Christentum eine entscheidende Rolle. Das Christentum weise zwar weit über Europa hinaus, es habe der europäischen Welt aber in Jahrhunderten ihre einheitliche Form gegeben. Die katholische Soziallehre — und damit die Überwindung der Klassengegensätze - wird zu einem Moment der europäischen Selbstbehauptung: das christliche Europa als eigenständige Lebensform zwischen Sozialismus und Kapitalismus. \\
Die europäische Integration hatte ihren Begriff gefunden.
Doch noch fühlten sich keineswegs alle politischen Kräfte dem Ziel der europäischen Einheit verbunden. Die Kommunisten etwa lehnten den Gedanken eines europäischen Zusammenschlusses ab; sie strebten vielmehr eine neue Kommunistische Internationale unter Führung Moskaus an. Die Führer des Widerstands in den befreiten Ländern Europas sahen ihre wichtigste Aufgabe im Wiederaufbau ihrer Länder. 'Europa' blieb auch nach dem 2. Weltkrieg nur eine Idee. Um wirkmächtig zu werden, bedurfte sie eines realpolitischen Anstoßes: Erst der kalte Krieg schuf die Rahmenbedingungen für die beginnende Integration (West-) Europas. \\ Die berühmt gewordene Züricher Rede von Winston Churchill, in der er im September 1946 forderte, 'etwas wie die Vereinigten Staaten von Europa zu schaffen', zog die Konsequenz aus einer Feststellung, die er nur wenige Monate zuvor getroffen hatte: In Europa sei 'ein eiserner Vorhang' niedergegangen. Die nun von ihm geforderte Einheit Europas sollte der Abwehr des sowjetischen Vordringens dienen. Ohne diese Wahrnehmung einer neuerlichen 'Bedrohung aus dem Osten', die der Westhälfte Deutschlands eine neue Bedeutung zuwies, wäre der Plan undenkbar.

\subsubsection{Der Weg nach Rom bis zur Europäischen Gemeinschaft}

In seiner Rede am 9. Mai 1950 stellt der französische Außenminister Robert Schuman den Plan vor, die Kohle- und Stahlproduktion Frankreichs und Deutschlands einer gemeinsamen Behörde zu unterstellen. Damit wurde der Grundstein für eine Europäische Union gelegt. Der 9. Mai ist seitdem der "Europatag". \\
Die sechs Länder Deutschland, Frankreich, Belgien, Niederlande, Luxemburg und Italien gründen die 'Europäische Gemeinschaft für Kohle und Stahl' (kurz EGKS). Der 1951 unterzeichnete Vertrag schafft den gemeinsamen Markt und die gemeinsame Kontrolle über Kohle und Stahl. \\
Im Jahr 1957 wurde im Zuge dessen die Europäische Wirtschaftsgemeinschaft (EWG) und die Europäische Atomgemeinschaft (EURATOM) gegründet auf Basis der 'Römischen Verträge'. Wieder dabei: Deutschland, Frankreich und die Benelux-Länder sowie Italien. Die EWG hat den internen Abbau von Zöllen und Handelshemmnissen zum Ziel. Die EURATOM will die friedliche Nutzung der Kernenergie und die gemeinsame Forschung gewährleisten, sowie die Sicherheitsvorschriften vereinheitlichen. Die Kommissionen der EWG und der EURATOM nehmen in Brüssel ihre Arbeit auf. Die Verhandlungen führten allgemein zu einer enormen Verbesserung des deutsch-französischen Verhältnisses. \\
Die unterschiedlichen Standorte für die verschiedenen Institutionen resultiert auch aus dieser Zeit. Die Frage des Sitzes war nämlich lange ein Streitpunkt zwischen den Mitgliedsstaaten. Unter der EGKS tagte die parlamentarische Versammlung in Straßburg. Die Hohe Behörde, der Rat und der Gerichtshof saßen in Luxemburg. Brüssel, wie genannt, wurde für EURATOM und EWG als Sitz für Rat und Kommission gewählt. \\
Am 7. Oktober 1958 wird in Luxemburg der Europäische Gerichtshof errichtet. Dieser sichert gemeinsam mit dem 'Gericht erster Instanz' die Wahrung des Rechts im gemeinschaftlichen Integrationsprozess. \\
Per Verordnung tritt die Gemeinsame Agrarpolitik (GAP) 1962 in Kraft. Die Ziele der GAP sind die Schaffung eines gemeinsamen Marktes für Agrarerzeugnisse und die finanzielle Solidarität in diesem Bereich (Mithilfe der Einrichtung eines Fonds). \\
Langfristig dachten die sechs Staaten schon an eine Zollunion. Sie wollten Handelshemmnisse abbauen und einen gemeinsamen Außenzoll ins Leben rufen. Zusätzlich sah der EWG-Vertrag vor, einen gemeinsamen Markt zu schaffen. Mit freiem Personen-, Dienstleistungs- und Kapitalverkehr. \\
Am 1. Juli 1967 wurden die Organe der drei Gemeinschaften (EGKS, EWG und EURATOM) zusammengelegt zur EG, der Europäischen Gemeinschaft. Für die EG wurde ein gemeinsamer Rat und eine gemeinsame Kommission eingesetzt. In mehreren Schritten gelang es, 1968 die Zölle zwischen den Mitgliedsstaaten abzubauen und einen gemeinsamen Außenzoll zu schaffen.

\subsubsection{Erster Zuwachs für die EG}

Mittlerweile war die westeuropäische Gemeinschaft auch für andere Länder attraktiv geworden: Dänemark, Irland und das Vereinigte Königreich von Großbritannien und Nordirland traten 1973 der EG bei. Anfang der 1970er Jahre gab es außerdem einen ersten Anlauf in Richtung Wirtschafts- und Währungsunion, der zunächst allerdings scheiterte. \\
Erst 1979 trat das EWS, das Europäische Währungssystem, in Kraft. Mit dem EWS erreichten die Staaten stabile Wechselkurse zwischen den beteiligten Währungen. Das Europäische Währungssystem war die Grundlage für die spätere Wirtschafts- und Währungsunion. \\
1979 war auch das Jahr des Europäischen Parlaments. Zum ersten Mal konnten die EG-Bürger die Abgeordneten für Europa direkt wählen. Und der Kreis der Wähler bekam wenige Jahre später weiteren Zuwachs: 1981 trat Griechenland der EG bei, 1986 Portugal und Spanien. \\
Man sprach vom "Europa der 12". Diese zwölf Mitgliedsstaaten machten 1986 einen weiteren wichtigen Schritt auf dem Weg zur Europäischen Union: In der Einheitlichen Europäischen Akte setzten sie sich einen gemeinsamen Binnenmarkt bis 1993 zum Ziel. \\
Die Jahre 1989 und 1990 veränderten Europa nachhaltig. Die Beendigung des Ost-West-Konflikts brachte der Europäischen Gemeinschaft neue Perspektiven und Aufgaben. Über die Wiedervereinigung wurde mit der DDR quasi über Nacht ein ehemals kommunistisches Land Mitglied der EG. Die EG leistete einen erheblichen Beitrag zur Entwicklung der Infrastruktur und der Wirtschaft in der ehemaligen DDR. Mit dem Ende des Kalten Kriegs waren im Prinzip auch schon die Weichen für eine Osterweiterung der Gemeinschaft gestellt. \\

\subsubsection{Europa am Ende des 20. Jahrhunderts}

Anfang der 1990er Jahre hatte die EG neuen Schwung bekommen. 1990 begann die erste Stufe der Wirtschafts- und Währungsunion (WWU). Damit begann ein Prozess in drei Stufen. Er sollte die Wirtschafts- und Währungspolitik der Mitgliedsstaaten unter einen Hut bringen und schließlich zu einer gemeinsamen Währung führen. In der ersten Stufe wurde der Kapitalverkehr zwischen den Staaten liberalisiert und die Zentralbanken arbeiteten stärker zusammen. \\
Planmäßig wurde am 1.1.1993 der EG-Binnenmarkt vollendet. Die EG war nun ein Wirtschaftsraum ohne Grenzen. Der Binnenmarkt gewährleistet seitdem den freien Verkehr von Personen, Waren, Dienstleistungen und Kapital. Damit wurde auch die alte Idee der EWG aus den 1950er Jahren Realität. \\
1993 trat auch der Maastrichter Vertrag von 1992 in Kraft und begründete schließlich die Europäische Union. Darin wurde die Kooperation in weiteren Politikbereichen vereinbart: in Angelegenheiten der Gemeinsamen Außen- und Sicherheitspolitik oder im Bereich Justiz und Inneres. Außerdem verständigten sich die Mitgliedsstaaten auf Abstimmungen bei Verbraucherschutz, Umweltfragen, Gesundheitswesen und Entwicklungshilfe. \\
Ein Jahr später begann die zweite Stufe der WWU. Eine gemeinsame Währung rückte näher. Die Mitgliedsstaaten verpflichteten sich beispielsweise, Preis- und Währung stabil zu halten und übermäßige öffentliche Defizite zu vermeiden. Gleichzeitig sollte der Aufbau einer Europäischen Zentralbank vorbereitet werden. Dazu wurde das Europäische Währungsinstitut in Frankfurt am Main errichtet. Die EZB setzt die stabilitätsorientierte Geldpolitik für das Euro-Währungsgebiet um. Sie bildet gemeinsam mit den nationalen Zentralbanken aller 15 Mitgliedstaaten der EU das "Europäische System der Zentralbanken (ESZB)". \\
Mitte der 1990er Jahre bekam die EU weiteren Zuwachs: 1995 traten Finnland, Schweden und Österreich bei. Im gleichen Jahr trat das Schengener Abkommen in Kraft zwischen Deutschland, Frankreich, Belgien, Niederlande, Luxemburg, Portugal und Spanien. Das Übereinkommen regelt die Durchführung und die Voraussetzungen, unter denen der freie Personenverkehr gewährleistet wird. Damit wurden unter anderem auch die Behandlung von Asylanträgen, die Einreise von Ausländern, Maßnahmen gegen Drogenhandel und die polizeiliche Zusammenarbeit geregelt. Das sogenannte Schengener Informationssystem sollte die grenzüberschreitende Verbrechensbekämpfung erleichtern. Später treten diesem Übereinkommen auch Italien, Griechenland, Dänemark, Finnland und Schweden bei. Österreich wendet das Abkommen seit 1998 in vollem Umfang an, im gleichen Jahr wurde auch der ständig wechselnde Vorsitz im Rat der Europäischen Union übernommen. Mit Norwegen, Island und der Schweiz als Nicht-Mitglieder bestehen als Schengen-Alternative Kooperationsabkommen. Die Inselstaaten Großbritannien und Irland sind zwar EU-Mitglieder, gehören jedoch nicht zum Schengen-Raum, sie setzten eine Ausnahmeregelung durch und führen weiterhin Kontrollen an ihren Grenzen durch. \\
Mit der Norderweiterung und dem Beitritt Österreichs hatte die Integration Westeuropas vorerst ihren (territorialen) Abschluss gefunden. Mit Ausnahme der Schweiz, Islands und Norwegens sowie einiger kleiner Fürstentümer oder Stadtstaaten war nun ganz Westeuropa unter dem Dach der Europäischen Union vereint. Nun begehrten langsam die früheren Ostblockstaaten Einlass. Wie in den Achtzigerjahren die Länder Südeuropas, so erhofften sich nun die mittel- und osteuropäischen Staaten von einem Beitritt politische Stabilisierung und wirtschaftliche Konsolidierung. \clearpage
Vor allem aber sollt die Aufnahme in die Union die endgültige Überwindung der Teilung Europas vollenden: Die einstigen Vasallen der Sowjetunion würden in die europäische Völkerfamilie zurückkehren. Nachdem der Europäische Rat 1993 in Kopenhagen die Tore der Union weit geöffnet hatte, stellten zwischen 1994 und 1996 viele der besprochenen Staaten Beitrittsanträge. Die 'Osterweiterung' der Union stand als Akt historischer Gerechtigkeit auf der Tagesordnung der europäischen Politik.

\subsubsection{Öffnung nach Osten}

1997 unterzeichneten die EU-Länder den Amsterdamer Vertrag, zwei Jahre später trat er in Kraft. Er sollte Europa auf das 21. Jahrhundert vorbereiten und so die Fortentwicklung der EU sichern. Darüber hinaus schrieb er Reformen der EU-Institutionen vor. Das Europaparlament bekam zum Beispiel mehr Rechte bei Mitentscheidungen. Die Stärkung der Gemeinsamen Außen- und Sicherheitspolitik und eine intensivere Kooperation im Bereich Justiz und Inneres wurden ebenfalls neu festgelegt. \\
Im bargeldlosen Zahlungsverkehr gab es den Euro schon 1999, da in diesem Jahr die dritte Stufe der Wirtschafts- und Währungsunion begann. Am 1. Januar 2002 war es dann soweit. In den Mitgliedsstaaten der Europäischen Währungsunion wurde das Euro-Bargeld zum alleinigen Zahlungsmittel. \\
Ende der 1990er Jahre verhandelte der Europäische Rat auch schon die nächsten Beitritte: 1998 zunächst mit Ungarn, Polen, Estland, der Tschechischen Republik, Slowenien und Zypern. 1999 auch mit Bulgarien, Lettland, Litauen, Malta, Rumänien und der Slowakei. \\
In dieser Zeit mussten auch lange beiseite geschobene Fragen beantwortet werden: Was war das Ziel des Integrationsprozesses? Welche Kompetenzen sollte die Union haben, welche bei den Nationalstaaten bleiben? \\
Im Dezember 2001 griff der Europäisch Rat im belgischen Laeken diese Diskussion auf. Fünfzig Jahre nach ihrer Gründung, so die Staats- und Regierungschefs in der 'Erklärung von Laeken über die Zukunft der Europäischen Union', befinde sich die Union an einem historischen Scheideweg. Die grundlegende Neuordnung des Kontinents durch die Osterweiterung der EU verlange nach 'einem anderen als dem vor fünfzig Jahren verfolgten Konzept'. Die EU müsse sich demokratisieren, sie müsse ihrer neuen Rolle in der globalisierten Welt gerecht werden und die an sie gerichteten Erwartungen der Bürger erfüllen. Die nächste Regierungskonferenz sollte Antworten auf eine ganze Fülle von Fragen geben. \\
Diese Regierungskonferenz sollte durch eine intensive Debatte über die Zukunft Europas vorbereitet werden. Nach den negativen Erfahrungen vorangeganger  Gipfel wie etwa der von Nizza, wo tage- und nächtelang lediglich halbherzige Kompromisse erarbeitet wurden, griffen die Mitgliedsstaaten damit auf ein Modell zurück, das sich schon einmal bewährt hatte: die Erarbeitung einer Charta der Grundrechte der Europäischen Union. \\
Aus den Verfassungsordnungen der Mitgliedsstaaten, der Rechtsprechung des EuGH und einzelner Bestimmungen der Verträge wurden die gemeinschaftlichen menschenrechtlichen Überzeugungen der Europäer festgelegt. So entstand die weltweit wohl modernste Menschenrechtserklärung, die neben traditionellen Freiheitsrechten auch wirtschaftliche und soziale Rechte sowie den Anspruch auf Datenschutz und genetische Selbstbestimmung festhält. \clearpage
Allerdings erhielt sie nur den Status einer Absichtserklärung, Rechtsgeltung kam ihr - zunächst jedenfalls - nicht zu. Ihr Anspruch war es freilich ohnehin, in erster Linie bereits geltende Rechtsüberzeugungen sichtbar zum Ausdruck zu bringen. Wenn schon nicht als eigenständige rechtliche Grundlage, so konnte sie dem EuGH doch als Hilfe bei der Interpretation des geltenden europäischen Rechts dienen. \\
Am 1. Mai 2004 traten dann zehn neue Mitgliedstaaten der EU bei: Estland, Lettland, Litauen, Tschechien, Slowakei, Slowenien, Ungarn, Polen, Malta und Zypern. 2007 folgten mit Bulgarien und Rumänien zwei weitere Länder, seit 2013 gehört auch Kroatien der EU an. Zu den zukünftigen Beitrittskandidaten gehören unter anderem die Türkei, Island und Serbien.

\subsubsection{Vertrag von Lissabon}

Durch den Vertrag von Lissabon wurde die Europäische Union institutionell reformiert. Das Ziel des Vertrages ist es, die EU demokratischer, transparenter und effizienter zu machen. Der Vertrag ist seit dem 1. Dezember 2009 in Kraft. \\
Schon vor der Erweiterung der Europäischen Union von 12 auf 15 Mitglieder Mitte der 1990er Jahre war klar, dass die EU sich einer institutionellen Reform unterziehen muss, um auch mit einer größeren Mitgliederzahl handlungsfähig zu bleiben. Da institutionelle Fragen jedoch Machtfragen sind, ist es weder durch den Vertrag von Amsterdam (1999 in Kraft getreten), noch durch den Vertrag von Nizza (seit 2003 gültig) gelungen, das Institutionengefüge der EU zu modernisieren. Ein weiterer Versuch, der Verfassungsvertrag, scheiterte im Jahr 2005 an negativen Referenden in den Niederlanden und in Frankreich. Der Lissabonner Vertrag ist nun der vierte Versuch, diese Aufgabe zu bewältigen. Auch seine Ratifizierung gestaltete sich nicht einfach, vor allem nachdem die Iren in einem ersten Referendum 2008 den Vertrag abgelehnt hatten. 2009 stimmten sie in einer zweiten Volksabstimmung für den Vertrag, sodass der Vertrag in Kraft treten konnte. \\
Durch den Lissabonner Vertrag vergrößert sich der Einfluss des Europäischen Parlaments, das (außer auf dem Feld der Außenpolitik) zu einem neben dem Rat der Europäischen Union gleichberechtigten Gesetzgeber wird (sog. Mitentscheidung). Auch die nationalen Parlamente erhalten mehr Einfluss. Sie werden früher über Vorschläge der Europäischen Kommission informiert und können diese schon während des Gesetzgebungsverfahrens zurückweisen, wenn sie den Grundsatz der Subsidiarität verletzt sehen. \\
Entscheidungen im Rat der Europäischen Union werden ab 2014 bzw. nach dem Auslaufen von Übergangsregelungen ab 2017 mit doppelter Mehrheit getroffen. Das bedeutet, dass jede Entscheidung der Zustimmung einer Mehrheit der Staaten (55 Prozent) bedarf, die gleichzeitig eine Mehrheit der Bevölkerung von 65 Prozent repräsentieren müssen. \\
Erstmals wird ein Europäisches Bürgerbegehren eingeführt, mit dem 1 Mio. Menschen aus verschiedenen Mitgliedstaaten die Europäische Kommission zwingen kann, sich mit einem Thema zu beschäftigen und einen Rechtsakt vorzuschlagen. \\
Die Kompetenzen zwischen EU und Mitgliedstaaten werden klarer und nachvollziehbarer geteilt. Sitzungen des Rates der Europäischen Union werden öffentlich sein, wenn der Rat gesetzliche Regelungen beschließt. \\
Die halbjährliche Rotation der Präsidentschaft wird auf der Ebene der Staats- und Regierungschefs ("Europäischer Rat") sowie der Außenminister abgeschafft. Der Europäische Rat wählt für 2 ½ Jahre eine Präsidentin oder einen Präsidenten. Den Vorsitz im Außenministerrat führt der Hohe Vertreter für die Außen- und Sicherheitspolitik, der zugleich Vizepräsident der Europäischen Kommission ist und über einen eigenen Europäischen Auswärtigen Dienst verfügt. 

\clearpage
\bibliographystyle{unsrt}
\bibliography{Pool1_Weinb_5BHIT}

\end{document}
