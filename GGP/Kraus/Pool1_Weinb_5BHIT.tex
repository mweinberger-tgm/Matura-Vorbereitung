\documentclass[letterpaper, 12pt]{article}

%%%%%%%%%%%%%%%%%%%%%%%%%%%%%
% DEFINITIONS
% Change those informations
% If you need umlauts you have to escape them, e.g. for an ü you have to write \"u
\gdef\mytitle{Ausarbeitung}
\gdef\mythema{Pool 1a}

\gdef\mysubject{GGP-Matura}
\gdef\mycourse{5BHIT 2015/16}
\gdef\myauthor{Michael Weinberger}

\gdef\myversion{1.0}
\gdef\mybegin{17. Mai 2016}
\gdef\myfinish{20. Mai 2016}

\gdef\mygrade{}
\gdef\myteacher{Betreuer: Kraus}
%
%%%%%%%%%%%%%%%%%%%%%%%%%%%%%

\input special/preamble.tex

\let\tempsection\section
\renewcommand\section[1]{\vspace{-0.3cm}\tempsection{#1}\vspace{-0.3cm}}
\WithSuffix\newcommand\section*[1]{\tempsection*{#1}}

\let\tempsubsection\subsection
\renewcommand\subsection[1]{\vspace{0cm}\tempsubsection{#1}\vspace{0cm}}

\let\tempsubsubsection\subsubsection
\renewcommand\subsubsection[1]{\vspace{0cm}\tempsubsubsection{#1}\vspace{0cm}}

\linespread{0.94}

\lhead{\mysubject}
\chead{}
\rhead{\bfseries\mythema}
\lfoot{\mycourse}
\cfoot{\thepage}
% Creative Commons license BY
% http://creativecommons.org/licenses/?lang=de
\rfoot{\ccby\hspace{2mm}\myauthor}
\renewcommand{\headrulewidth}{0.4pt}
\renewcommand{\footrulewidth}{0.4pt}

\begin{document}
\parindent 0pt
\parskip 6pt

\pagenumbering{Roman} 
\input{special/title}

\clearpage
\thispagestyle{empty}
\tableofcontents

\newpage
\pagenumbering{arabic}
\pagestyle{fancy}

%\vspace{-0.5cm}

\section{Pool 1a - Europa im Wandel \cite{buch}}

\subsection{Entstehung und Entwicklung der EU}

\subsubsection{Der Europagedanke in der Nachkriegszeit}

Ein Jahr vor
Kriegsende, im Frühjahr und Sommer 1944, arbeiteten Widerstandsgruppen aus neun europäischen Staaten, darunter
Deutschland, eine gemeinsame Erklärung aus. Sie beriefen sieh
auf die Ziele, die 1941 in der Atlantik—Charta vom britischen
Premierminister Winston Churchill und dem amerikanischen
Präsidenten Franklin D. Roosevelt wurden waren: Demokratie, soziale Gerechtigkeit und Achtung der Menschenrechte. Diese Ziele aber könnten nur erreicht werden, wenn die verschiedenen Länder der Welt sich bereit erklären, das Dogma der absoluten Staatssouveränität abzustreifen, wie sie definierten. \\
Da der Schlüssel zum Frieden in Europa liege, sei zunächst eine mit starken Institutionen ausgestattete europäische 'Bundesordnung' vonnöten, an die sämtliche außen- und sicherheitspolitischen Kompetenzen abzutreten seien. Des Weiteren spielt das Christentum eine entscheidende Rolle. Das Christentum weise zwar weit über Europa hinaus, es habe der europäischen Welt aber in Jahrhunderten ihre einheitliche Form gegeben. Die katholische Soziallehre — und damit die Überwindung der Klassengegensätze - wird zu einem Moment der europäischen Selbstbehauptung: das christliche Europa als eigenständige Lebensform zwischen Sozialismus und Kapitalismus. \\
Die europäische Integration hatte ihren Begriff gefunden.
Doch noch fühlten sich keineswegs alle politischen Kräfte dem Ziel der europäischen Einheit verbunden. Die Kommunisten etwa lehnten den Gedanken eines europäischen Zusammenschlusses ab; sie strebten vielmehr eine neue Kommunistische Internationale unter Führung Moskaus an. Die Führer des Widerstands in den befreiten Ländern Europas sahen ihre wichtigste Aufgabe im Wiederaufbau ihrer Länder. 'Europa' blieb auch nach dem 2. Weltkrieg nur eine Idee. Um wirkmächtig zu werden, bedurfte sie eines realpolitischen Anstoßes: Erst der kalte Krieg schuf die Rahmenbedingungen für die beginnende Integration (West-) Europas. Die berühmt gewordene Züricher Rede von Winston Churchill, in der er im September 1946 forderte, 'etwas wie die Vereinigten Staaten von Europa zu schaffen', zog die Konsequenz aus einer Feststellung, die er nur wenige Monate zuvor getroffen hatte: In Europa sei 'ein eiserner Vorhang' niedergegangen. Die nun von ihm geforderte Einheit Europas sollte der Abwehr des sowjetischen Vordringens dienen. Ohne diese Wahrnehmung einer neuerlichen 'Bedrohung aus dem Osten', die der Westhälfte Deutschlands eine neue Bedeutung zuwies, wäre der Plan undenkbar.

\clearpage
\bibliographystyle{unsrt}
\bibliography{Pool1_Weinb_5BHIT}

\end{document}
