\documentclass[letterpaper, 12pt]{article}

%%%%%%%%%%%%%%%%%%%%%%%%%%%%%
% DEFINITIONS
% Change those informations
% If you need umlauts you have to escape them, e.g. for an ü you have to write \"u
\gdef\mytitle{Ausarbeitung}
\gdef\mythema{Ethik}

\gdef\mysubject{Systemtechnik-Matura}
\gdef\mycourse{5BHIT 2015/16}
\gdef\myauthor{Michael Weinberger}

\gdef\myversion{1.0}
\gdef\mybegin{21. April 2016}
\gdef\myfinish{01. Juni 2016}

\gdef\mygrade{}
\gdef\myteacher{Betreuer: Graf/Borko}
%
%%%%%%%%%%%%%%%%%%%%%%%%%%%%%

\input special/preamble.tex

\let\tempsection\section
\renewcommand\section[1]{\vspace{-0.3cm}\tempsection{#1}\vspace{-0.3cm}}
\WithSuffix\newcommand\section*[1]{\tempsection*{#1}}

\let\tempsubsection\subsection
\renewcommand\subsection[1]{\vspace{0cm}\tempsubsection{#1}\vspace{0cm}}

\let\tempsubsubsection\subsubsection
\renewcommand\subsubsection[1]{\vspace{0cm}\tempsubsubsection{#1}\vspace{0cm}}

\linespread{0.94}

\lhead{\mysubject}
\chead{}
\rhead{\bfseries\mythema}
\lfoot{\mycourse}
\cfoot{\thepage}
% Creative Commons license BY
% http://creativecommons.org/licenses/?lang=de
\rfoot{\ccby\hspace{2mm}\myauthor}
\renewcommand{\headrulewidth}{0.4pt}
\renewcommand{\footrulewidth}{0.4pt}

\begin{document}
\parindent 0pt
\parskip 6pt

\pagenumbering{Roman} 
\input{special/title}

\clearpage
\thispagestyle{empty}
\tableofcontents

\newpage
\pagenumbering{arabic}
\pagestyle{fancy}

%\vspace{-0.5cm}
\textbf{Kompetenzen für den Teilbereich 'Ethische Aspekte, Rechtliche Grundlagen und Gesellschaftliche Auswirkungen der Informationstechnologie'} \newline
\begin{itemize}
	\item können die Interaktion zwischen Informationstechnik, Gesellschaft und Politik analysieren und in ihrer Arbeit beachten
	\item kennen rechtliche Grundlagen der Informationstechnologe und reflektieren Auswirkungen des Missbrauchs von Informationstechnologien
	\item beachten ethische Standards bei Datensicherheit, Datenschutz und Privatsphäre
	\item beachten ethische Grundwerte in der Sicherheits- und Überwachungstechnik
\end{itemize}

\clearpage

\section{Cloud Computing und Internet of Things}

\subsection{Einführung}

In den letzten Jahren geht der Trend immer mehr in Richtung Cloud Computing, dem Anbieten von verschiedensten IT-Dienstleistungen über das Netzwerk. Dieser Prozess funktioniert dynamisch und an den Bedarf des Nutzers angepasst. Die Grundlage der Cloud bietet das Internet als Plattform. Darüber werden Verbindungen zu externen Servern hergestellt, um Anwendungen bereitzustellen. Damit ist auch die Möglichkeit zur Datenspeicherung mit inbegriffen. \\
Der Benutzer muss die verschiedenen IT-Services (Software, Plattformleistungen, Infrastrukturleistungen) nicht mehr selber bereitstellen, sondern mietet diese kurzerhand. Die Anbieter bieten hier ein flexibles und dienstbasiertes Geschäftsmodell (Everything as a Service). Dieser Service ist nicht nur besser skalierbar und ausfallsicherer, sondern im Zweifelsfall auch billiger als ein großer Server-Verbund im eigenen Netzwerk. \\
Man unterscheidet zwischen öffentlichen und privaten Clouds. Die Public Cloud wird von der breiten Öffentlichkeit genutzt, sprich verschiedene Kunden. Die Private Cloud ist hiervon das Gegenteil, das Angebot ist auf einen bestimmten Kunden ausgerichtet und auf dessen interne Anwendungen beschränkt. \\
Als zweiter großer Bereich dieses Pools ist Internet of Things definiert. Internet of Things bezeichnet die Vernetzung von Gegenständen mit dem Internet, damit diese Gegenstände selbstständig über das Internet kommunizieren und so verschiedene Aufgabe für den Besitzer erledigen können. Der Anwendungsbereich erstreckt sich dabei von einer allgemeinen Informationsversorgung über automatische Bestellungen bis hin zu Warn- und Notfallfunktionen. Als Beispiel, die Waschmaschine stellt fest, dass zu wenig Waschmittel im Haus ist, und bestellt dieses von selbst nach. Oder auch selbstfahrende Autos, die untereinander kommunizieren und so Staubildungen bestmöglich verhindern sollen. \cite{cloud, iot}

\subsection{Bedenken}

Es ist zu bedenken, dass hier die (persönlichen) Daten auf fremden Servern liegen. Der Begriff Cloud ist für viele sehr abstrakt, und nicht näher definiert, wo die Daten schlussendlich landen. Es ist jedoch so, dass diese lediglich auf einem anderen Server landen, der von fremden Personen bzw. Unternehmen verwaltet wird und deren Handlungsspielraum unterliegt. Die Privatsphäre kann durch diese Vollmacht sehr einfach gebrochen werden, was auch zu einem Missbrauch der Daten führen kann. Ebenso ist fraglich (in AGBs meist definiert, aber für User unbekannt), ob die Daten in der Cloud auch an Dritte weitergegeben werden, beispielsweise zu Werbezwecken. \\
Durch Internet of Things kann ein besonders gutes Profil über einen Menschen erstellt werden. Diese Gegenstände, beispielsweise Fitnessarmbänder, können ständig den Gesundheitszustand mitloggen. Der gläserne Mensch wird dadurch immer mehr Realität, da, wie vorhin besprochen, nie restlos sichergestellt werden kann, wo diese Daten dann schlussendlich landen. Hier ist der sogenannte Begriff 'Data Mining' eine gute Erklärung, möglichst viele Daten generieren über einen Benutzer, um durch Algorithmen Zusammenhänge zu finden und Rückschlüsse auf das Verhalten zu ziehen.

\subsection{Formulierung ethischer Fragestellungen}

diskutieren anhand der bedenken auf grundlage der theorie \newline
aufstellen einiger wichtiger fragen und versuch einer ethischen urteilsfindung

\clearpage

\section{Automatisierung, Regelung und Steuerung}

Themengebiet wird ausgelassen (1 von 1)

\clearpage

\section{Security, Safety, Availability}

\subsection{Einführung}

Begriffserklärungen, aufkommende internetkriminalität, anwendungsfälle, beispiel, rechtliche grundlagen + auswirkungen des missbrauchs, welche arten der it-kriminalität gibt es, wie kann man sich absichern? \newline availability --> it-gesellschaft kontext herstellen, welche services sind (überlebens-?) wichtig, wirtschaftlicher schaden durch manipulation/downtime

\subsection{Bedenken}

sammeln von fakten \newline
datendiebstahl, hacking, ... sicherheits- und überwachungstechnik (auch kameras, massenüberwachung, vorratsdatenspeicherung), verlust der privatsphäre durch totale sicherheit, wie viele daten darf ein unternehmen von mir sammeln, social engineering als zugang zu sensiblen daten, mensch = schwachstelle

\subsection{Formulierung ethischer Fragestellungen}

diskutieren anhand der bedenken auf grundlage der theorie \newline
aufstellen einiger wichtiger fragen und versuch einer ethischen urteilsfindung

\clearpage

\section{Authentication, Authorization, Accounting}

\subsection{Einführung}

begriffserklärungen, anwendungsfälle, beispiel, hilft die identität zu bestätigen und zu wahren

\subsection{Bedenken}

sammeln von fakten \newline
wegfallen der anonymität, wie unsicher ist mein passwort, hacking, identitätsdiebstahl, wie sicher ist mein system / sicherheitslücken

\subsection{Formulierung ethischer Fragestellungen}

diskutieren anhand der bedenken auf grundlage der theorie \newline
aufstellen einiger wichtiger fragen und versuch einer ethischen urteilsfindung

\clearpage

\section{Disaster Recovery}

\subsection{Einführung}

begriffserklärungen, anwendungsfälle, beispiel, notwendigkeit hervorheben, gutes level mit hohen kosten verbunden

\subsection{Bedenken}

sammeln von fakten \newline
leaks, whistleblowing bis hin zu wikileaks, panama papers als folge schlechter disasterplanung oder übersehen von sicherheitslücken, datenverlust für unternehmen nicht tragbar, wie weit hafte ich für datenverlust

\subsection{Formulierung ethischer Fragestellungen}

diskutieren anhand der bedenken auf grundlage der theorie \newline
aufstellen einiger wichtiger fragen und versuch einer ethischen urteilsfindung

\clearpage

\section{Algorithmen und Protokolle}

\subsection{Einführung}

begriffserklärungen, anwendungsfälle, beispiel, notwendigkeit hervorheben, verschlüsselung, sicherheitszertifizierungen, was ist standard in der it, vertrauensproblem bei ungesicherten verbindungen, rechtslage bis von zu gesetzen zum verbot von verschlüsselung

\subsection{Bedenken}

sammeln von fakten \newline
sicherheitslücken in verschlüsselungsalgorithmen, unverschlüsselte kommunikation kann abgefangen werden, schüre ich durch verschlüsselung terrorismus? wenn alle daten verschlüsselt sind, können verbrechen geplant werden und behörden haben keine handhabe.

\subsection{Formulierung ethischer Fragestellungen}

diskutieren anhand der bedenken auf grundlage der theorie \newline
aufstellen einiger wichtiger fragen und versuch einer ethischen urteilsfindung

\clearpage

\section{Konsistenz und Datenhaltung}

\subsection{Einführung}

begriffserklärungen, anwendungsfälle, beispiel, notwendigkeit hervorheben, erklärung inkonsistenzen, wieso datenbanken, kundendaten konsisten sein sollen, fehlertoleranz, schutz vor missbrauch oder verlust (lost update bei bankdaten?)

\subsection{Bedenken}

sammeln von fakten \newline
wo liegt die verantwortung? kunde, der daten freigibt oder unternehmer, der daten ablegt. was passiert bei datendiebstählen? habe ich die vollständige einsicht, was auf den servern abgelegt ist?

\subsection{Formulierung ethischer Fragestellungen}

diskutieren anhand der bedenken auf grundlage der theorie \newline
aufstellen einiger wichtiger fragen und versuch einer ethischen urteilsfindung

\clearpage

\bibliographystyle{unsrt}
\bibliography{SYT_Ethik_Weinberger}
\lstlistoflistings
\listoffigures

\end{document}
