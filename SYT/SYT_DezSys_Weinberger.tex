\documentclass[letterpaper, 12pt]{article}

%%%%%%%%%%%%%%%%%%%%%%%%%%%%%
% DEFINITIONS
% Change those informations
% If you need umlauts you have to escape them, e.g. for an ü you have to write \"u
\gdef\mytitle{Feingliederung}
\gdef\mythema{DezSys}

\gdef\mysubject{Systemtechnik-Matura}
\gdef\mycourse{5BHIT 2015/16}
\gdef\myauthor{Michael Weinberger}

\gdef\myversion{1.0}
\gdef\mybegin{21. April 2016}
\gdef\myfinish{\today}

\gdef\mygrade{}
\gdef\myteacher{Betreuer: Graf/Borko}
%
%%%%%%%%%%%%%%%%%%%%%%%%%%%%%

\input special/preamble.tex

\let\tempsection\section
\renewcommand\section[1]{\vspace{-0.3cm}\tempsection{#1}\vspace{-0.3cm}}
\WithSuffix\newcommand\section*[1]{\tempsection*{#1}}

\let\tempsubsection\subsection
\renewcommand\subsection[1]{\vspace{0cm}\tempsubsection{#1}\vspace{0cm}}

\let\tempsubsubsection\subsubsection
\renewcommand\subsubsection[1]{\vspace{0cm}\tempsubsubsection{#1}\vspace{0cm}}

\linespread{0.94}

\lhead{\mysubject}
\chead{}
\rhead{\bfseries\mythema}
\lfoot{\mycourse}
\cfoot{\thepage}
% Creative Commons license BY
% http://creativecommons.org/licenses/?lang=de
\rfoot{\ccby\hspace{2mm}\myauthor}
\renewcommand{\headrulewidth}{0.4pt}
\renewcommand{\footrulewidth}{0.4pt}

\begin{document}
\parindent 0pt
\parskip 6pt

\pagenumbering{Roman} 
\input{special/title}

\clearpage
\thispagestyle{empty}
\tableofcontents

\newpage
\pagenumbering{arabic}
\pagestyle{fancy}

%\vspace{-0.5cm}
\textbf{Kompetenzen für Dezentrale Systeme}

\begin{itemize}
	\item	\textbf{Lastenverteilung auf Applikationsebene} \newline
	\textit{'können Lastverteilung auf Applikationsebene realisieren'}
	\item 	\textbf{Sicherheitskonzepte} \newline
	\textit{'können Sicherheitskonzepte für verteilte, dezentrale Systeme entwickeln'}
	\item 	\textbf{Durchführung von Transaktionen in verteilten Systemen} \newline
	\textit{'können in dezentralen Systemen Transaktionen durchführen'}
	\item	\textbf{Programmiertechniken zur Realisierung von entfernten Prozeduren, Methoden und Objekten} \newline
	\textit{'können Programmiertechniken in verteilten Systemen zur Realisierung von entfernten Prozeduren, Methoden und Objekten anwenden sowie webbasierte Dienste und Messaging-Dienste in solchen Systemen implementieren'}
\end{itemize}

\clearpage

\section{Cloud Computing und Internet of Things}

\subsection{Einführung}

cc/iot --> Was versteht man darunter

\subsection{Realisierung von entfernten Prozeduren, Methoden, Objekten zur Interkommunikation}

ipc, rpc, java rmi

\subsection{Grundlagen Messaging-Dienste}

mom, ...

\subsection{Anwendung mit webbasierten Diensten}

JEE, REST Grundlagen, Frameworks, ...

\clearpage

\section{Automatisierung, Regelung und Steuerung}

Themengebiet wird ausgelassen (1 von 1)

\clearpage

\section{Security, Safety, Availability}

\subsection{Grundlegende Sicherheitskonzepte}

Intrusion Detection, Honey Pot, Application Firewall, ... sowie Unterschiede

\subsection{Risiken \& Gefahren von dezentralen Systemen}

Beschreibung und Lösungsansätze

\subsection{Frameworks}

Funktionsweisen, verschiedene Ansätze

\clearpage

\section{Authentication, Authorization, Accounting}

\subsection{Beschreibung der Grundlagen}

Auth, Aut, Audit

\subsection{Was ist LDAP?}
Erklärung der Funktionsweise

\subsection{Benutzerverwaltung mit LDAP}

Serverarten, Zugriff darauf

\subsection{Möglichkeiten zur Implementierung/alternative verteilte Authentifizierungsdienste}

KDC, kerberos, Single Sign On, ...

\clearpage

\section{Disaster Recovery}

\subsection{Backup-Strategien}

Häufigkeit, wo werden sie aufbewahrt, sicherheitslevels

\subsection{Disaster Recovery Plan}

Aufstellen, was sollte drin sein, ...

\subsection{Best Practice für die vorgegebenen Anforderungen}

hot standby, cold standby, cluster

\subsection{Rollback}

Wie wird zentrales Image verteilt?

\clearpage

\section{Algorithmen und Protokolle}

\subsection{Techniken zur Prüfung und Erhöhung der Sicherheit von dezentralen Systemen}

Vorgehensweisen, symm. Verschlüsselung, SSL/TLS-Protokoll, ...

\subsection{Grundlagen Lastverteilung}

Hervorheben der Notwendigkeit, erste ansätze

\subsection{Load Balancing-Algorithmen}

round robin, weighted distr, least conn, ...

\subsection{Session-basiertes Load Balancing}

Erklärung der Vorteile und Idee zur Verwirklichung

\subsection{Load-Balancing-Frameworks}

Einführung und Unterschiede

\clearpage

\section{Konsistenz und Datenhaltung}

\subsection{Grundlagen \& Erklärung}

cap-theorem, hervorheben der notwendigkeit, vermeidung von inkonsistenzen, ...

\subsection{Verschiedene Transaktionsprotokolle}

2-phase, 3-phase, 2-phase-lock, long-duration, transaction, jta

\subsection{Lösungsansätze bei Transaktionskonflikten}

Aufzählen und erklären der verschiedenen vorgehensweisen

\clearpage

\clearpage
\bibliographystyle{unsrt}
\bibliography{SYT_DezSys_Weinberger}
\lstlistoflistings
\listoffigures

\end{document}
