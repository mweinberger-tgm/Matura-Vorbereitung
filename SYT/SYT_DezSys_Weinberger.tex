\documentclass[letterpaper, 12pt]{article}

%%%%%%%%%%%%%%%%%%%%%%%%%%%%%
% DEFINITIONS
% Change those informations
% If you need umlauts you have to escape them, e.g. for an ü you have to write \"u
\gdef\mytitle{Ausarbeitung}
\gdef\mythema{DezSys}

\gdef\mysubject{Systemtechnik-Matura}
\gdef\mycourse{5BHIT 2015/16}
\gdef\myauthor{Michael Weinberger}

\gdef\myversion{1.0}
\gdef\mybegin{21. April 2016}
\gdef\myfinish{16. Mai 2016}

\gdef\mygrade{}
\gdef\myteacher{Betreuer: Graf/Borko}
%
%%%%%%%%%%%%%%%%%%%%%%%%%%%%%

\input special/preamble.tex

\let\tempsection\section
\renewcommand\section[1]{\vspace{-0.3cm}\tempsection{#1}\vspace{-0.3cm}}
\WithSuffix\newcommand\section*[1]{\tempsection*{#1}}

\let\tempsubsection\subsection
\renewcommand\subsection[1]{\vspace{0cm}\tempsubsection{#1}\vspace{0cm}}

\let\tempsubsubsection\subsubsection
\renewcommand\subsubsection[1]{\vspace{0cm}\tempsubsubsection{#1}\vspace{0cm}}

\linespread{0.94}

\lhead{\mysubject}
\chead{}
\rhead{\bfseries\mythema}
\lfoot{\mycourse}
\cfoot{\thepage}
% Creative Commons license BY
% http://creativecommons.org/licenses/?lang=de
\rfoot{\ccby\hspace{2mm}\myauthor}
\renewcommand{\headrulewidth}{0.4pt}
\renewcommand{\footrulewidth}{0.4pt}

\begin{document}
\parindent 0pt
\parskip 6pt

\pagenumbering{Roman} 
\input{special/title}

\clearpage
\thispagestyle{empty}
\tableofcontents

\newpage
\pagenumbering{arabic}
\pagestyle{fancy}

%\vspace{-0.5cm}
\textbf{Kompetenzen für Dezentrale Systeme}

\begin{itemize}
	\item	\textbf{Lastenverteilung auf Applikationsebene} \newline
	\textit{'können Lastverteilung auf Applikationsebene realisieren'}
	\item 	\textbf{Sicherheitskonzepte} \newline
	\textit{'können Sicherheitskonzepte für verteilte, dezentrale Systeme entwickeln'}
	\item 	\textbf{Durchführung von Transaktionen in verteilten Systemen} \newline
	\textit{'können in dezentralen Systemen Transaktionen durchführen'}
	\item	\textbf{Programmiertechniken zur Realisierung von entfernten Prozeduren, Methoden und Objekten} \newline
	\textit{'können Programmiertechniken in verteilten Systemen zur Realisierung von entfernten Prozeduren, Methoden und Objekten anwenden sowie webbasierte Dienste und Messaging-Dienste in solchen Systemen implementieren'}
\end{itemize}

\clearpage

\section{Cloud Computing und Internet of Things}

\subsection{Einführung}

\subsubsection{Was versteht man unter Cloud Computing?}

Heutzutage setzen viele Hersteller auf Cloud Computing und bieten dementsprechende Plattformen an, der Trend geht immer mehr in diese Richtung. \begin{center}
Konkret geht es bei Cloud Computing um die Auslagerung von Anwendungen, Daten und Rechenvorgängen ins Web. \\
\end{center} 
Dies könnte zum Beispiel die Auslagerung von Bürosoftware wie Tabellenkalkulation, Textverarbeitung oder CRM-Systemen in die Cloud sein.
Diese Auslagerung bietet einige Vorteile. Die Synchronisation zwischen mehreren Rechnern wird nicht mehr relevant und gemeinsame Arbeit an Dokumenten durch die zentrale Ablage vereinfacht. Für das Absichern der Datensätze ist die Cloud Computing-Plattform und die Anwendung selbst verantwortlich. Natürlich ist auch der Datenspeicher in der 'Wolke' begrenzt, jedoch in jeder Hinsicht größer als der eines einzelnen Rechners oder Festplattenverbundes im normalen Stil. Verglichen mit traditionellen Systemumgebungen sind Cloud Computing-Plattformen wesentlich einfacher zu verwalten. Der Grund dafür ist der hohe Abstraktionsgrad der Plattformen, denn um typische Administrationsaufgaben wie Load Balancing oder Serverwartung kümmert sich der Anbieter. Bei Rechenvorgängen ist der Vorteil einer besseren Skalierbarkeit gegeben, dass man auf einen großen Pool von Instanzen zurückgreifen kann. Dank Cloud Computing und dessen hoher Flexibilität kann man diese Server beispielsweise auch für wenige Stunden oder Tage, in denen sie benötigt werden, \textit{on demand} mieten und somit Betriebskosten einsparen. Die Bereitstellung erfolgt innerhalb von Minuten, und kommt ohne komplexe Verträge aus.\cite{ausarbeitungcc}

\subsubsection{Was versteht man unter IoT?}

Das Internet der Dinge (Internet of Things / IoT) ist ein Gebilde, bei dem Objekte, Tiere oder Menschen mit einem einzigartigen Identifikator ausgestattet sind. Weiterhin ist damit die Möglichkeit verbunden, Daten über ein Netzwerk ohne Interaktionen Mensch-zu-Mensch oder Mensch-zu-Computer zu übertragen. Ein Ding im Internet der Dinge kann zum Beispiel eine Person mit einem Herzschrittmacher, ein Nutztier auf einem Bauernhof mit einem Biochip-Transponder oder ein Automobil mit eingebauten Sensoren sein. Letzteres könnte eine Warnung auslösen, wenn der Reifendruck zu niedrig ist. Im Prinzip ist jedes vom Menschen geschaffene Objekt ein Kandidat, das sich mit einer IP-Adresse ausstatten lässt und Daten via Netzwerk übertragen kann. Bisher wurde das Internet der Dinge am häufigsten mit Maschine-zu-Maschine-Kommunikation bei einer Fertigungsstraße in Verbindung gebracht. Sind Produkte mit M2M-Kommunikation ausgestattet, werden sie häufig als \textit{intelligent} oder \textit{smart} bezeichnet. Der durch IPv6 wesentlich größere Adressraum ist ein wichtiger Faktor bei der Entwicklung des Internets der Dinge. Durch die wachsende Anzahl an intelligenten Knoten (Nodes) erwartet man, dass es neue Bedenken für Privatsphäre, Datenhoheit und Sicherheit gibt. \cite{iot} \clearpage

\subsection{Realisierung von entfernten Prozeduren, Methoden, Objekten zur Interkommunikation}

\subsubsection{Sockets, Interprozesskommunikation}

vorstufe von rmi, für einfachere sachen

\subsubsection{RPC}

\subsubsection{Java RMI}

\subsubsection{CORBA}

ipc, rpc, java rmi

\subsection{Grundlagen Messaging-Dienste}

mom, ...

\subsection{Anwendung mit webbasierten Diensten}

JEE, REST Grundlagen, Frameworks, ...

\clearpage

\section{Automatisierung, Regelung und Steuerung}

Themengebiet wird ausgelassen (1 von 1)

\clearpage

\section{Security, Safety, Availability}

\subsection{Grundlegende Sicherheitskonzepte}

Intrusion Detection, Honey Pot, Application Firewall, ... sowie Unterschiede

\subsection{Risiken \& Gefahren von dezentralen Systemen}

Beschreibung und Lösungsansätze

\subsection{Frameworks}

Funktionsweisen, verschiedene Ansätze

\clearpage

\section{Authentication, Authorization, Accounting}

\subsection{Beschreibung der Grundlagen}

Auth, Aut, Audit

\subsection{Was ist LDAP?}
Erklärung der Funktionsweise

\subsection{Benutzerverwaltung mit LDAP}

Serverarten, Zugriff darauf

\subsection{Möglichkeiten zur Implementierung/alternative verteilte Authentifizierungsdienste}

KDC, kerberos, Single Sign On, ...

\clearpage

\section{Disaster Recovery}

\subsection{Backup-Strategien}

Häufigkeit, wo werden sie aufbewahrt, sicherheitslevels

\subsection{Disaster Recovery Plan}

Aufstellen, was sollte drin sein, ...

\subsection{Best Practice für die vorgegebenen Anforderungen}

hot standby, cold standby, cluster

\subsection{Rollback}

Wie wird zentrales Image verteilt?

\clearpage

\section{Algorithmen und Protokolle}

\subsection{Techniken zur Prüfung und Erhöhung der Sicherheit von dezentralen Systemen}

Vorgehensweisen, symm. Verschlüsselung, SSL/TLS-Protokoll, ...

\subsection{Grundlagen Lastverteilung}

Hervorheben der Notwendigkeit, erste ansätze

\subsection{Load Balancing-Algorithmen}

round robin, weighted distr, least conn, ...

\subsection{Session-basiertes Load Balancing}

Erklärung der Vorteile und Idee zur Verwirklichung

\subsection{Load-Balancing-Frameworks}

Einführung und Unterschiede

\clearpage

\section{Konsistenz und Datenhaltung}

\subsection{Grundlagen \& Erklärung}

cap-theorem, hervorheben der notwendigkeit, vermeidung von inkonsistenzen, ...

\subsection{Verschiedene Transaktionsprotokolle}

2-phase, 3-phase, 2-phase-lock, long-duration, transaction, jta

\subsection{Lösungsansätze bei Transaktionskonflikten}

Aufzählen und erklären der verschiedenen vorgehensweisen

\clearpage

\clearpage
\bibliographystyle{unsrt}
\bibliography{SYT_DezSys_Weinberger}
\lstlistoflistings
\listoffigures

\end{document}
